\documentclass[aip,jcp,preprint,superscriptaddress,showpacs,preprintnumbers,amsmath,amssymb]{revtex4-1}
%\documentclass{article}

\setcitestyle{super}

%\usepackage[a4paper, top=0.9in, bottom=1 in]{geometry}
\usepackage{graphicx}
\usepackage{amsmath}
\usepackage{amsthm}
\usepackage{mathrsfs}

\renewcommand{\thesection}{\Alph{section}}
\renewcommand{\thesubsection}{\thesection.\arabic{subsection}}

\numberwithin{equation}{section}
\renewcommand{\theequation}{\thesection\arabic{equation}}

\renewcommand{\thetable}{\Roman{table}}

\newcommand{\vct}[1]{\mathbf{#1}}
\providecommand{\vr}{} % clear \vr
\renewcommand{\vr}{\vct{r}}
\newcommand{\vk}{\vct{k}}
\newcommand{\vR}{\vct{R}}
\newcommand{\dvk}{\frac{d\vk}{(2\pi)^D}}
\newcommand{\tdvk}{\tfrac{d\vk}{(2\pi)^D}}





\begin{document}





\title{Supplemental Material I}
\maketitle





\section{Fourier transform}





The Fourier transform of a spherical-symmetric function is
%
%
%
\begin{align}
  \tilde A(k)
&\equiv
  \int A(r) \, e^{-i\vk \cdot \vr} \, d\vr
  \notag \\
&=
  \frac{ (2\pi)^{D/2} }
       { k^{D/2-1} }
  \int_0^R A(r) \, J_{D/2-1}(k \, r) \, r^{D/2} \, dr,
\label{eq:fftsphr}
\end{align}
%
%
%
where $J_\nu(k \, r)$ is the Bessel function
of order $\nu \equiv D/2-1$,
and $R$ is the radial cutoff beyond which $A(r)$ vanishes.





\subsection{Odd dimensions}





For an odd dimension, $D$,
the Bessel function can be replaced
by the spherical Bessel function\cite{arfken}
$j_\delta(x) \equiv \sqrt{\pi/(2x)} J_{\delta + 1/2}(x)$,
where $\delta \equiv (D - 3)/2$ is an integer,
and
%
%
%
\begin{align}
\tilde A(k)
\equiv
\frac{ 2 (2 \pi)^{\delta + 1} }
     { k^\delta }
\int_0^R
A(r) j_\delta(k r) r^{\delta + 2} \, dr.
\label{eq:fftsphrj}
\end{align}
%
%
%
The advantage is that the spherical Bessel functions are
linear combinations of $\sin x$ and $\cos x$
with rational coefficients,
e.g.,
$j_1(x) = x^{-2} \sin x - x^{-1} \cos x$.
%
Thus,
Eq. \eqref{eq:fftsphrj} can be implemented as
a combination of
a few fast sine and cosine transforms\cite{press}.
%
For high dimensions,
the transforms involve large inverse powers of $x$,
and thus need high-precision floating-point numbers
at small $x$.





\subsection{Even dimensions}





In even dimensions,
we can view Eq. \eqref{eq:fftsphr} as a Hankel transform:
%
%
%
\begin{align*}
  \hat \alpha(k)
=
  \int_0^R \alpha(r) \, J_\nu(k r) \, r \, dr,
\end{align*}
%
%
%
where
%$\nu \equiv D/2 - 1$,
$\alpha(r) \equiv A(r) \, r^\nu$,
and
$\hat \alpha(k) \equiv \tilde A(k) \, k^\nu \, (2\pi)^{-\nu-1}$.
%
The discrete Hankel transform is
evaluated as a quadrature sampled at
$r_k = (\gamma_{\nu,k} / \gamma_{\nu,M}) \, R$\cite{
lado1967, johnson1987, *lemoine1994}
as
%
%
%
\begin{align*}
\tilde \alpha(k_m)
\approx
\left(
  \frac{ R }
  { \gamma_{\nu, M} }
\right)^2
\,
\sum_{k = 1}^{M - 1}
  \alpha(r_k) \,
  J_\nu\left(
    \frac{ \gamma_{\nu, m} \, \gamma_{\nu, k} }
         { \gamma_{\nu, M} }
  \right)
  \frac{ 2 }
       { \left[ J_{\nu+1}(\gamma_{\nu, k}) \right]^2 },
\end{align*}
%
%
%
where $k_m = \gamma_{\nu,m}/R$,
and $\gamma_{\nu, m}$ is the $m$th positive zero
of the Bessel function $J_\nu(z)$.
%
Unlike the previous case,
this Hankel transform requires $O(M^2)$ space and time,
and we used fewer grid points in even dimensions.





\subsection{Grid spacing}





To reduce the numerical error caused
by the hard-core discontinuity,
we choose the grid spacing of the integration such that
the hard-core boundary, $r = 1$, lies at the midpoint of two
consecutive grid points.
%
Under this constraint,
the radial cutoff $R$ is adjusted to be slightly greater than
$n_{\max} + 2$,
where $n_{\max}$ is the maximal order of the virial expansion.
%
The choice of R, however,
could be larger for some purposes.
%
The parameters are listed in Table \ref{tab:paramft}.



\begin{table}
\begin{tabular}{ l l l l }
\hline
$D$
&
Method$^\dagger$
&
Precision (bits)$^\ddagger$
&
$M^\S$
\\ \hline
%
$3, 5, \dots, 13$
&
FFT
&
113 (quadruple)
&
$2^{21} = 2097152$
\\
%
$15, 17, \dots, 27$
&
FFT
&
256
&
$2^{18} = 262144$
\\
%
$29$
&
FFT
&
384
&
$2^{18} = 262144$
\\ \hline
%
$2, 4, 6, \dots, 22$
&
DHT
&
53 (double)
&
$2^{18} = 262144$
\\
%
$2, 4, 6, \dots, 30$
&
DHT
&
64 (extended)
&
$2^{18} = 131072$
\\ \hline
\multicolumn{4}{p{\linewidth}}{
$^\dagger$
FFT: fast Fourier transform\cite{press}
(the FFTW\cite{frigo2005} library was used when possible);
DHT: discrete Hankel transform\cite{
johnson1987, *lemoine1994}.
} \\
\multicolumn{4}{p{\linewidth}}{
$^\ddagger$
Number of bits to represent the significant of
the floating-point number.
The MPFR library\cite{fousse2007}
was used for precisions greater than 113.
} \\
\multicolumn{4}{p{\linewidth}}{
$^\S$
Maximal number of grid points along $r$.
} \\ \hline
%
\end{tabular}
\caption{
Parameters of computing Fourier transforms.
}
\label{tab:paramft}
\end{table}





\section{Extrapolation}





Our numerical integrations depend nontrivially
on the number of grid points, $M$, along $r$.
%
This dependency can be reduced
by an extrapolation to infinite $M$\cite{
adda-bedia2008a, *adda-bedia2008b}.
%
Practically, we fit the computed virial coefficients
to the following formula:
%
%
%
\begin{equation}
B(h)
=
B(\infty)
+
\epsilon_2 \, h^2
+
\epsilon_4 \, h^4
+
\cdots,
\label{eq:error_series}
\end{equation}
%
%
%
where $h \equiv R/M$
is the average grid spacing,
then use $B(\infty)$ as the best estimate.
%
Note that the absence of odd powers of $h$
results from the Euler-Maclaurin formula\cite{
arfken, whittaker}
in the uniform-grid case.



In practice, we collected at least four data points
with different spacings.
%
We took the first threes terms from \eqref{eq:error_series},
and used regression to estimate
$B(\infty)$, $\epsilon_2$, and $\epsilon_4$.
%
The data points were weighted by $h^{-2}$ in the fitting
to increase the weight of higher resolution data.
%
To be conservative,
we also computed $B'(\infty)$ from fitting the data points
with the three smallest spacings, $h$.
%
Then,
we set the error of $B$ to the larger of
%
$\left| B'(\infty) - B(\infty) \right|$
%
and the uncertainty of $B(\infty)$ from the fitting.





\section{\label{sec:series}
Transformation of series}





Here we compute the density expansion of
the cavity distribution function
for a general closure.
%
We assume a local relation between $y(\vr)$ and $t(\vr)$:
$y(\vr) = Y[t(\vr)]$,
and thus omit the ``$(\vr)$'' below.
%
Then, the problem is to find the coefficients $y_l$
in the expansion $y = \sum_l y_l \, \rho^l$,
given $t = \sum_l t_l \, \rho^l$ at a given $\vr$.
%
This is done by algebraically transforming series\cite{
kilpatrick1969, *aldrovandi1980},
as exemplified below.



First,
given two series
$a = \sum_l a_l \, \rho^l$
and
$b = \sum_l b_l \, \rho^l$,
the linear combination is given by
$\alpha \, a + \beta \, b =
\sum_l (\alpha \, a_l + \beta \, b_l) \, \rho^l$.
%
For the product,
we have
$(a \cdot b)_l = \sum_{i = 0}^l a_i \, b_{l - i}$.
%
For the quotient
$a / b = \sum_l d_l \, \rho^l$,
we invert the product formula by the recursion as
%
$d_l =
\left.
  \left( a_l - \sum_{i = 1}^l b_i \, d_{l-i} \right)
\middle/
  b_0
\right.$
%
(assuming $b_0 \ne 0$),
with $l$ increased successively from 0.



Consider now the $\nu$th power:
$a^\nu = p \equiv \sum_l p_l \, \rho^l$
(assuming $a_0 \ne 0$).
%
Noting that
$a \, (\partial_\rho p) = \nu \, p \, (\partial_\rho a)$,
and comparing the coefficients of $\rho^{l - 1}$
on both sides yields
$p_0 = a_0^\nu$,
and for $l \ge 1$,
%
%
%
\begin{equation}
p_l
=
\sum_{j = 0}^{l - 1}
  \left.
    [\nu \, l - (\nu + 1) \, j] \, p_j \, a_{l - j}
  \middle/
    (l \, a_0)
  \right.
\label{eq:series_pow}
\end{equation}
%
%
%
Now if the cavity distribution function $y$
can be expressed as a power series of $t$ as
$y = Y(t) = \sum_{\nu = 0} Y_\nu \, t^\nu$,
then each term can be expanded by Eq. \eqref{eq:series_pow}.



Some special function forms of $Y(t)$, however,
permit faster recursions.
%
The components of the exponential function $y = \exp t$,
for example, are given by
$y_0 = \exp t_0$,
and for $l \ge 1$,
%
\begin{equation}
y_l
=
\sum_{j = 1}^l
  \left.
    y_{l - j} \, j \, t_j
  \middle/
    l
  \right..
\label{eq:series_expt}
\end{equation}
%
Conversely,
if we know the coefficients $y_l$,
the expansion of the logarithm,
$t \equiv \log y$,
can be found from solving Eq. \eqref{eq:series_expt} for $t_l$.
%
Thus,
$t_0 = \log y_0$,
and for $l \ge 1$,
%
%
%
\begin{equation}
t_l
=
\frac{ 1 } { l \, y_0 }
\left[
  l \, y_l
  -
  \sum_{j = 1}^{l - 1}
    j \, t_j \, y_{l - j}
\right].
\label{eq:series_logy}
\end{equation}
%
%
%
Besides,
the geometric series $b = \sum_{n = 0} a^n = 1/(1-a)$
can be expanded by the division formula as
%
$b_l =
\left.
  \left[ \delta_{l0} + \sum_{j = 1}^l a_j \, b_{l-j}\right]
\middle/
  (1 - a_0)
\right.$
%
(assuming $a_0 \ne 1$).



The above recipes serve as building blocks
for more complex relations.
%
For example,
to expand the power $p = a^b$
with a polynomial exponent
$b = \sum_{j = 0}^{+\infty} b_j \, \rho^j$
[which is an intermediate step in
solving the $\lambda$-DSC equation
based on the HC closure\cite{
hutchinson1971, *hutchinson1972}:
$y = (1 + s t)^{1/s}$],
%
we compute first the logarithm
$\log p = b \, \log a$,
and then the exponential
$\exp(\log p)$.
%
Similarly,
for the BPGG closure\cite{ballone1986},
the expansion of
$y = \exp\left[ (1 + s \, t)^{1/s} - 1 \right]$
is computed from a power series
followed by an exponentiation.



Finally, consider a closure given inversely as
%
%
%
\begin{equation}
t = T(y).
\label{eq:series_definv}
\end{equation}
%
%
%
The Rowlinson 1\cite{rowlinson1965}
[$t = \eta \log y + (1 - \eta) \, (y-1)$]
and Hurst\cite{hurst1965}
($t = y^m \, \log y$)
closures are of this type.
%
In such a case,
we can obtain $y_l$
order by order starting from $l = 1$.
%
We first define a partial sum
$y^{(l)} \equiv \sum_{j=0}^l y_j \, \rho^j$.
%
Using this in Eq. \eqref{eq:series_definv} yields
%
%
%
\[
t_l
=
\left[ T( y^{(l)} ) \right]_l
=
\left[ T( y^{(l-1)} ) \right]_l
+
( \Delta T )_l,
\]
%
%
%
where
$( \Delta T )_l$
is the contribution of $y_l \, \rho^l$
to the coefficient of $\rho^l$,
and can thus be written as $C \, y_l$.
%
Here,
$C$ depends on $y^{(l - 1)}$ only through
the constant term $y_0$,
for the coupling of $y_l \, \rho^l$
with other terms,
$y_1 \, \rho$, $y_2 \, \rho^2$, \dots,
leads to a power higher than $\rho^l$.
%
So,
\[
(\Delta T)_l
=
\lim_{ \rho \rightarrow 0 }
  \frac{ T(y_0 + y_l \rho^l) - T(y_0) }
       { \rho^l }
=
T'(y_0) \, y_l,
\]
%
where $T' \equiv dT/dy$,
and
$y_0 = 1$ here.
%
Thus,
\[
y_l
=
\frac{ t_l - \left[ T(y^{(l - 1)}) \right]_l }
     { T'(1) }.
\]



Take Hurst's closure
%
%$t = y^m \log y$,
%
for example,
we have $T'(1) = 1$,
and
$y_l = t_l - \left[ (y^{(l-1)})^m \log y^{(l-1)} \right]_l$.
%
Note that in this example,
$y$ can also be explicitly inverted as
$y =
1 +
\left[
  \sum_{k = 1} \left. (1 - k m)^{k-1} \, t^k \middle/ k! \right.
\right]$
or
$\log y =
\sum_{k = 1} \left. (-k m)^{k-1} \, t^k \middle./ k! \right.$
by Lagrange inversion\cite{whittaker, uhlenbeck1962}.





\section{YBG equation}





The YBG equation\cite{
yvon1935, *born1946, *nijboer1952, hill, *rice1965, hansen}
is
%
\begin{align*}
&
\nabla_1
  \left[
    \log g(\vr_{12})
    +
    \beta \, \phi(\vr_{12})
  \right]
\\
&
=
-\beta \rho \int
  [ \nabla_1 \phi(\vr_{13}) ] \,
  g(\vr_{13}) \, g(\vr_{32}) \, d\vr_3.
\end{align*}
%
It can be rewritten as
%
\begin{equation}
\nabla_1 \omega(\vr_{12})
=
\rho \int
  [ \nabla_1 f(\vr_{13}) ] \,
  y(\vr_{13}) \, h(\vr_{32}) \, d\vr_3,
\label{eq:ybg}
\end{equation}
%
where
$\omega(\vr) \equiv \log y(\vr) = \log g(\vr) + \beta \, \phi(\vr)$.
%
Equation \eqref{eq:ybg} is structurally similar to the OZ relation.
%
For a spherical-symmetric potential,
it can be written in the Fourier space as
%
\begin{equation}
i \vk \tilde \omega(k)
=
i \vk \rho \, \tilde \gamma(k) \, \tilde h(k),
\label{eq:ybg_omegak1}
\end{equation}
%
where,
$i \vk \tilde \omega(k)$
and
$i \vk \tilde \gamma(k)$
are the Fourier transformations of
$\nabla \omega (\vr)$
and
$(\nabla f) \, y(\vr)$,
respectively, and
%
%
%
\begin{align}
\tilde \gamma(k)
&=
\frac{ 1 } { k }
\frac{ \partial } { \partial k }
\int
  \frac{ f'(r) } { r }
  y(r) \, e^{-i \vk \cdot \vr} \, d\vr
\notag \\
&=
-\left( \frac{ 2 \pi } { k } \right)^{D/2}
  \int f'(r) \, y(r) \, J_{D/2}(k \, r) \, r^{D/2} \, dr.
\label{eq:ybg_gammak1}
\end{align}
%
%
%
Here, we have used Eq. \eqref{eq:fftsphr} in the second step.



For the hard-sphere potential,
we have
$f(r) = -\Theta(1 - r)$,
where $\Theta(\cdots)$ denotes the unit step function,
i.e., $f(r)$ is $-1$ for $r < 1$, or $0$ otherwise.
%
So,
$f'(r) = \delta(r - 1)$
and
$\tilde f(k) = -(2\pi/k)^{D/2} \, J_{D/2}(k)$
by Eq. \eqref{eq:fftsphr}.
%
It follows that
we can rewrite Eq. \eqref{eq:ybg_gammak1} as
$\tilde \gamma(k) = y(1) \, \tilde f(k)$,
and
Eq. \eqref{eq:ybg_omegak1} as
$\tilde \omega(k) = \rho \, y(1) \, \tilde f(k) \, \tilde h(k)$,
whose density components satisfy
%
%
%
\begin{equation}
\tilde \omega_l(k)
=
\sum_{j = 0}^{l - 1}
\tilde f(k) \, y_j(1) \, \tilde h_{l-1-j}(k).
\label{eq:ybg_omegakl}
\end{equation}
%
%
%
The inverse Fourier transform, $\omega_l(r)$,
can then be used to compute $y_l(r)$ from Eq. \eqref{eq:series_expt}.
%
Overall, we follow the iteration
%
\begin{align*}
&
\tilde \omega_l
  \xrightarrow{ \mathrm{F.T.} }
\omega_l
  \xrightarrow{ \mathrm{Eq.\,\eqref{eq:series_expt},\,with\;} t \rightarrow \omega }
\\
&
y_l
  \xrightarrow{ h_l = (1 + f) \, y_l }
h_l
  \xrightarrow{ \mathrm{F.T.} }
\tilde h_l
  \xrightarrow{ \mathrm{Eq.\,\eqref{eq:ybg_omegakl} } }
\tilde \omega_{l + 1}.
\end{align*}



The virial- and cavity-route virial coefficients are computed by
%
%
%
\begin{equation}
B_n^{(v)}
=
\frac{ 1 } { 2 D }
\int
  r \, f'(r) \, y_{n - 2}(r) \, d\vr,
\label{eq:Bn_virial}
\end{equation}
%
%
%
and
%
%
%
\begin{equation}
B_n^{(y)}
=
\frac{ n - 1 } { n }
\left[
  \log y(\vct 0)
\right]_{n - 1},
\label{eq:Bn_cavity}
\end{equation}
%
%
%
respectively.
%
For the compressibility route,
we use
$\partial_\rho
\left(
  \beta \, P^{(c)}
\right)
=
\left[
  1 + \rho \int h(\vr) \, d\vr
\right]^{-1}$,
%
and
%
\begin{equation}
B_n^{(c)}
=
\frac{ 1 } { n }
\left\{
  \left[
    1 + \rho \int h(\vr) \, d\vr
  \right]^{-1}
\right\}_{n - 1},
\label{eq:Bn_compressibility}
\end{equation}
%
which can be computed using Eq. \eqref{eq:series_pow}
with $\nu = -1$.





\section{\label{sec:kirkwood}Kirkwood equation}





The Kirkwood equation\cite{
kirkwood1935, *hoover1962, *stell1962,
hill, *rice1965} is
%
%
%
\begin{align}
&
\log g(\vr_{12}; \xi)
+
\beta \, \phi(\vr_{12}; \xi)
\notag \\
&
=
-\beta \, \rho \,
\int_0^\xi
d\xi'
\int \partial_{\xi'} \phi(\vr_{13}; \xi') \,
  g(\vr_{13}; \xi') \, g(\vr_{32}) \, d\vr.
\label{eq:kirkwood_original}
\end{align}
%
%
%
Here,
particle 1 is distinguished from the rest,
and its interaction with other particles,
$\phi(\vr_{1i}; \xi)$
is tuned on by the charging parameter $\xi$,
such that
$\phi(\vr_{1i}; \xi = 0) = 0$,
and
$\phi(\vr_{1i}; \xi = 1) = \phi(\vr_{1i})$
is the normal pair potential.
%
Thus,
we omit $\xi$ for quantities unrelated to particle 1.



Equation \eqref{eq:kirkwood_original}
can be rewritten as
%
%
%
\begin{equation}
\omega(\vr_{12}; \xi)
=
\rho
\int
  K(\vr_{13}; \xi) \, h(\vr_{32}) \, d\vr_3,
\label{eq:kirkwood}
\end{equation}
%
%
%
where
$\omega(\vr; \xi) \equiv \log y(\vr; \xi)
=
\log g(\vr; \xi) + \beta \, \phi(\vr; \xi)$,
and
%
%
%
\begin{equation}
K(\vr; \xi)
=
\int_0^\xi
d\xi' \,
\partial_{\xi'} f(\vr; \xi') \,
y(\vr; \xi').
\label{eq:kirkwood_K}
\end{equation}



Below we choose an $f$-bond that varies linearly with $\xi$:
%
\[
  f(\vr; \xi) = \xi \, f(\vr),
\]
%
which is equivalent to setting
$\phi(\vr; \xi) =
\left.
  -\log[1 + \xi \, f(\vr)]
\middle/
  \beta
\right.$.
%
Since the solution generally
depends on the choice of $\xi$,
i.e., how the interaction of particle 1 with other particles
is turned on,
our solution is only one of infinitely many solutions.
%
Then, we expand the correlation functions
as series of both $\rho$ and $\xi$ as
%
$\omega(\vr; \xi)
=
\sum_{l = 1}^{+\infty}
\sum_{m = 1}^l
\omega_{l,m}(\vr) \, \rho^l \, \xi^m$,
%
$y(\vr; \xi)
=
\sum_{l = 0}^{+\infty}
\sum_{m = 0}^l
y_{l, m}(\vr) \, \rho^l \, \xi^m$,
%
and
$K(\vr; \xi)
=
\sum_{l = 0}^{+\infty}
\sum_{m = 1}^{l + 1}
K_{l, m}(\vr) \, \rho^l \, \xi^m$,
%
where $y_{l, 0}(\vr) = \delta_{l, 0}$,
and
%
$K_{l, 1}(\vr) = f(\vr) \, \delta_{l, 0}$.
%
The expansion in $\rho$ and $\xi$
has a graphical interpretation
(cf. Section \ref{sec:cr_mayer_graphs}):
the indices $l$ and $m$ indicate
the number of non-root vertices
and the number of vertices adjacent to the first root,
respectively.



Using these expansions in Eq. \eqref{eq:kirkwood_K} yields
%
\begin{equation}
K_{l,m}(\vr)
=
\left.
  f(\vr) \, y_{l,m-1}(\vr)
\middle/
  m
\right.,
\label{eq:kirkwood_Klm}
\end{equation}
%
and in the Fourier-space version of
Eq. \eqref{eq:kirkwood} yields
%
\begin{equation}
\tilde \omega_{l,m}(\vk)
=
\sum_{j = 0}^{l - 1}
\tilde K_{l - j - 1, m}(\vk) \,
\tilde h_j(\vk).
\label{eq:kirkwood_omegaklm}
\end{equation}
%
Then,
the inverse Fourier transform $\omega_{l,m}(\vr)$
can be used to compute $y_{l,m}(\vr)$
by an extension of Eq. \eqref{eq:series_expt}:
%
%
%
\begin{equation}
y_{l,m}(\vr)
=
\sum_{ j = 1 }^l
\sum_{ k = \max\{1, m - l + j\} }^{ \min\{j, m\} }
\left.
  y_{l-j, m-k}(\vr) \, j \, \omega_{j, k}(\vr)
\middle/
  l
\right..
\label{eq:kirkwood_ylm}
\end{equation}
%
%
%
Finally, $h_l(\vr)$ is computed as
%
%
%
\begin{equation}
h_l(\vr)
=
\left\{
  [1 + f(\vr)] \,
  \sum_{m = 0}^l y_{l,m}(\vr)
\right\}
- \delta_{l,0}.
\label{eq:kirkwood_hl}
\end{equation}



The overall iteration is the following
%
%
%
\begin{align*}
&
\tilde\omega_{l, m = 1, \dots, l}
\xrightarrow{ \mathrm{F.T.} }
\omega_{l, m = 1, \dots, l}
\xrightarrow{ \mathrm{Eq.\,\eqref{eq:kirkwood_ylm}} }
\\
&
y_{l, m = 1, \dots, l}
\xrightarrow{ \mathrm{Eqs.\,\eqref{eq:kirkwood_Klm},\,
                            \eqref{eq:kirkwood_hl} } }
\left(
K_{l, m = 1, \dots, l+1}, h_l
\right)
\xrightarrow{ \mathrm{F.T.} }
\\
&
\left(
\tilde K_{l, m = 1, \dots, l+1}, \tilde h_l
\right)
\xrightarrow{ \mathrm{Eq.\,\eqref{eq:kirkwood_omegaklm}} }
\tilde \omega_{l+1, m = 1, \dots, l+1}.
\end{align*}
%
%
%
The virial coefficients of
the compressibility, virial, and cavity routes
can be obtained from Eqs.
\eqref{eq:Bn_compressibility},
\eqref{eq:Bn_virial},
and
\eqref{eq:Bn_cavity},
respectively.





\section{\label{sec:extension}
Generalizations of the DSC closure}





\subsection{Symmetry of the DSC equation}





We can generalize the DSC equation in the main text
to any pair of closures,
designated by ``$0$'' and ``$1$'',
$y^0(\vr) \equiv y^0[t(\vr)]$
and
$y^1(\vr) \equiv y^1[t(\vr)]$:
%
%
%
\begin{equation}
y(\vr)
=
y^0(\vr)
+
\sum_{l = 2}^{\infty}
y_l \, w_l(\vr) \, \rho^l,
\label{eq:yr_wl_series}
\end{equation}
%
%
%
and
%
%
%
\begin{equation}
w_l(\vr)
=
\left[
y^1(\vr) - y^0(\vr)
\right]_l.
\label{eq:wl_y1_y0}
\end{equation}



This equation
produces correlation functions symmetric
to the closures $0$ and $1$.
%
That is, if we replace
Equation \eqref{eq:yr_wl_series}
by one based on closure $1$:
%
%
%
\[
y(\vr)
=
y^1(\vr)
-
\sum_{l = 2}^{+\infty}
\lambda_l^* \, w_l(\vr) \, \rho^l,
\]
%
%
%
the resulting correlation functions
$c_l(\vr)$, $t_l(\vr)$, and $y_l(\vr)$
would be the same as those obtained from
Eq. \eqref{eq:yr_wl_series},
and $\lambda_l + \lambda_l^* = 1$,
under the DSC condition:
%
%
%
\begin{equation}
B_{l+2}^{(c)}
=
B_{l+2}^{(v)}.
\label{eq:Bc_eq_Bv}
\end{equation}
%
%
%
This can be shown by an induction on $l$.





\subsection{Exponential expansion}





The above linear correction can be replaced
by an exponential one:
%
\begin{align*}
y(\vr)
=
y^0(\vr) \,
\exp\left[
\sum_{l = 2}^{+\infty} \lambda_l \, d_l(\vr) \, \rho^l
\right],
\end{align*}
%
where
%
\begin{align*}
d_l(\vr)
=
\left[
  \log y^1(\vr)
  -
  \log y^0(\vr)
\right]_l
\end{align*}
%
For example,
if closures $0$ and $1$ are the HNC and PY, respectively,
then
%
$d_l(\vr) =
\left\{
  \log \left[ 1 + t(\vr) \right]
\right\}_l - t_l(\vr)$,
%
and the density expansion of the logarithm
can be computed by Eq. \eqref{eq:series_logy}.





\subsection{$\lambda$-DSC equation}





We now consider the $\lambda$-DSC equation.
%
We wish to determine the coefficients $\lambda_l$
in the expansion:
%
\begin{align*}
\lambda =
\lambda_2
+
\lambda_3 \, \rho
+
\lambda_4 \, \rho^2
+ \cdots.
%\label{eq:DSC_lambda}
\end{align*}
%
By expanding both sides of the closure,
%
\begin{equation}
y(\vr) =
y^0(\vr) + \lambda \,
\left[ y^1(\vr) - y^0(\vr) \right],
\label{eq:yr_linear01}
\end{equation}
%
as series of the density, we get
%
\begin{equation*}
y_l(\vr)
=
y_l^0(\vr)
+
\sum_{j = 2}^l
\lambda_{l - j + 2} \, w_j(\vr),
\end{equation*}
%
where,
$w_j(\vr)$ is defined in Eq. \eqref{eq:wl_y1_y0}.
%
The sum starts from $j = 2$,
for we have assumed
the correct first two density components of
the cavity distribution functions:
%
$y_0^0 = y_0^1(\vr) = 1$
%
$y_1^0 = y_1^1(\vr) = (f*f)(\vr)$.
%
This condition is satisfied by
all cavity distribution functions listed
in Table I of main text.



For each order $l = n - 2$,
imposing Eq. \eqref{eq:Bc_eq_Bv} yields
%
\begin{equation}
\lambda_l =
-\frac{ B_{l+2}^{(c, \mathrm{uncorr})}
      - B_{l+2}^{(v, \mathrm{uncorr})} }
      { \Delta B_{l+2}^{(c)}
      - \Delta B_{l+2}^{(v)} },
\label{eq:lambdadsc_lambda}
\end{equation}
%
the same form as the DSC one
but with modified definitions:
%
%
%
\begin{equation}
B_n^{(c, \mathrm{uncorr})}
\equiv
-\frac{1}{n}
\int
\left\{
  \left[ f(\vr) + 1 \right] \, y_{n-2}^{(-)}(\vr)
  - t_{n-2}(\vr)
\right\}
\, d\vr,
\label{eq:lambdadsc_Bc}
\end{equation}
%
\begin{equation}
B_n^{(v, \mathrm{uncorr})}
\equiv
\frac{1}{2D}
\int
\vr \cdot \nabla f(\vr) \, y_{n-2}^{(-)} \, d\vr,
\label{eq:lambdadsc_Bv}
\end{equation}
%
\begin{equation}
\Delta B_n^{(c)}
\equiv
-\frac{1}{n}
\int
\left[ f(\vr) + 1 \right] \, w_2(\vr) \, d\vr,
\label{eq:lambdadsc_dBc}
\end{equation}
%
\begin{equation}
\Delta B_n^{(v)}
\equiv
\frac{1}{2D}
\int
\vr \cdot \nabla f(\vr) \, w_2(\vr) \, d\vr,
\label{eq:lambdadsc_dBv}
\end{equation}
%
with
%
\[
y_{n-2}^{(-)}(\vr)
\equiv
y_{n-2}^0(\vr)
+
\sum_{j=3}^{n-2}
\lambda_{n-j} \, w_j(\vr).
\]



The above solution recovers that of the DSC equation
under the substitutions
$y_{n-2}^{(-)}(\vr) \rightarrow y_{n-2}^0(\vr)$
in Eqs. \eqref{eq:lambdadsc_Bc} and \eqref{eq:lambdadsc_Bv},
and
$w_2(\vr) \rightarrow w_{n-2}(\vr)$
in Eqs. \eqref{eq:lambdadsc_dBc} and \eqref{eq:lambdadsc_dBv}.
%
Note that the effective correction function
for determining $\lambda_{n-2}$ is now $w_2(\vr)$
instead of $w_{n-2}(\vr)$,
which enters the denominator of
the right-hand side of Eq. \eqref{eq:lambdadsc_lambda}.
%
Unlike $w_{n-2}(\vr)$,
$w_2(\vr) \propto (f*f)^2(\vr)$ is short-ranged.
%
Thus,
we expect the $\lambda$-DSC equation
to break down in high orders,
especially rapidly in high dimensions.

If, instead of Eq. \eqref{eq:yr_linear01},
$y(\vr)$ depends nonlinearly on $\lambda$,
we modify the above solution as follows.
%
Assuming the closure can be written as
$y(\vr) = Y[t(\vr); \lambda]$,
we can always separate the contribution of $\lambda_l$ as
%
\begin{equation}
y_l(\vr)
=
y_l^{(-)}(\vr)
+
w_2(\vr) \, \lambda_l,
\end{equation}
%
where
%
\begin{equation}
y_l^{(-)}(\vr)
\equiv
\left[
  Y\left(
    \sum_{i=1}^{l-1} t_i \, \rho^i; \,
    \sum_{j=2}^{l-1} \lambda_j \, \rho^{j-2}
  \right)
\right],
\end{equation}
%
where can be evaluated using techniques
in Section \ref{sec:series}, and
%
\[
w_2(\vr) \equiv
\lim_{\rho \rightarrow 0}
\left(
  \frac{ 1 } { \rho^2 }
  \frac{ \partial Y } { \partial \lambda }
\right).
\]
%
Equations \eqref{eq:lambdadsc_lambda}-\eqref{eq:lambdadsc_dBv}
remain valid.
%
Take the HC closure\cite{
hutchinson1971, *hutchinson1972}
for example,
$\lambda$ is the parameter $s$,
and
$y = Y(t; s) = 1 + t + (1-s)\,t^2/2 + \cdots$.
%
So
$w_2 =
\lim_{\rho \rightarrow 0}
(1/\rho^2)(\partial Y/\partial s)
= - t_1^2 / 2 = -(f*f)^2/2$.





\section{\label{eq:dcorr}
Expansions of correlation functions around a finite density}





Here, we study an extension of the method
to expand correlation functions
around a finite (nonzero) density.
%
In this way,
we can study a system at a higher density than
the radius of convergence of the virial series.
%
This technique can also be used to study the system
around a certain density of interest.
%
If this density happens to be zero,
we recover the method in the main text.



We start by expanding
an arbitrary correlation function, $A(\vr)$,
which can be $c(\vr)$, $t(\vr)$, $h(\vr)$, or $y(\vr)$,
around density $\rho$ as
%
\[
\Bigl. A(\vr) \Bigr|_{\rho + \Delta \rho}
=
\sum_{l = 0}^{+\infty}
A_{l;\rho}(\vr) \, \Delta \rho^l,
\]
%
where
%
$(A)_{l;\rho} \equiv
\left.
  \left(
    \left.
      \partial^l A
    \middle/
      \partial \rho^l
    \right|_\rho
  \right)
\middle/
  l!
\right.$
%
denotes the coefficient of $\Delta \rho^l$,
and
$A_{0\;\rho}(\vr) \equiv A(\vr)$.
%
Using the expansions in the Ornstein-Zernike (OZ) relation,
\[
\left.
  1 + (\rho + \Delta\rho) \, \tilde h(\vk)
\right|_{\rho + \Delta \rho}
=
\left[
  \Bigl.
    1 - (\rho + \Delta\rho) \, \tilde c(\vk)
  \Bigr|_{\rho + \Delta \rho}
\right]^{-1},
\]
we get, for $l = 0$,
\[
1 + \rho \, \tilde h
=
(1 - \rho \, \tilde c)^{-1}
\]
[dropping ``$(\vk)$'' for convenience],
and for $l \ge 1$,
%
%
%
\begin{align*}
\rho \, \tilde h_{l;\rho}
&=
\left(
  \left[
    1 - \rho \, \tilde c
    - \sum_{i=1}^l (\rho \, \tilde c_{i;\rho}
                  + \tilde c_{i-1;\rho}) \, \Delta \rho^i
  \right]^{-1}
\right)_{l;\rho}
- \tilde h_{l-1;\rho}
\\
&=
K \,
\sum_{i = 1}^l
(\tilde h_{l - 1 - i; \rho}
+ \rho \, \tilde h_{l - i; \rho})
\,
(\tilde c_{i-1; \rho}
+ \rho \, \tilde c_{i; \rho})
- \tilde h_{l-1; \rho},
\end{align*}
%
%
%
where $K = 1 + \rho \, \tilde h$,
and $\tilde h_{-1;\rho} = 1$.
%
On the second line,
we have used the recurrence relation
for the inverse (cf. Section \ref{sec:series}).
%
It can be shown by induction
that the above result is equivalent to
%
\begin{equation*}
\tilde h_{l;\rho}
=
K \, \tilde h_{l-1;\rho} \, \tilde c_{0;\rho}
+
K \sum_{i=1}^l
  (\tilde h_{l-1-i;\rho} + \rho \, \tilde h_{l-i;\rho}) \,
  \tilde c_{i;\rho}.
%\label{eq:oz_hl_finiterho}
\end{equation*}
%
The indirect correlation is obtained from
$\tilde t_{l;\rho}
=
\tilde h_{l;\rho}
-
\tilde c_{l;\rho}.$



For the closure, we have,
%
\begin{align*}
c_{l;\rho}(\vr)
&=
[1 + f(\vr)] \, y_{l;\rho}(\vr)
- t_{l;\rho}(\vr) - \delta_{l,0},
\\
y_{l;\rho}(\vr)
&=
Y\left[
\sum_{i = 0}^l t_{i;\rho}(\vr) \, \Delta \rho^i
\right]_l,
\end{align*}
%
where
the density expansion of $y(\vr) = Y[t(\vr)]$
is to be computed by the techniques
in Section \ref{sec:series}.

For $l = 1$,
the explicit first derivatives
$\partial_\rho A(\vr) \equiv A_{1;\rho}(\vr)$
(for $A = c$, $t$ and $y$)
are given below:
%
%
%
\begin{equation}
\begin{split}
\partial_\rho \tilde t(\vk)
&=
\tilde h^2(\vk)
+
\rho \, \tilde h(\vk) \,
[2 + \rho \, \tilde h(\vk)] \, \partial_\rho \tilde c(\vk),
\\
%
\partial_\rho c(\vr)
&=
[1 + f(\vr)] \, \partial_\rho y(\vr)
- \partial_\rho t(\vr),
\\
\partial_\rho y(\vr)
&=
Y'[t(\vr)] \, \partial_\rho t(\vr).
\end{split}
\label{eq:dcorl}
\end{equation}
%
%
%
Once we have computed
$c(\vr)$, $t(\vr)$, and $y(\vr)$,
$\partial_\rho c(\vr)$,
$\partial_\rho t(\vr)$,
$\partial_\rho y(\vr)$
can be solved by iterating Eq. \eqref{eq:dcorl}.
%
The solution determines the virial-route
$\partial_\rho \left( \beta \, P^{(v)} \right)$
without numerical differentiation:
%
%
%
\[
\partial_\rho (\beta \, P^{(v)})
=
\frac{ 2 \, \beta \, P^{(v)} } { \rho } - 1
+ \frac{ \rho^2 } { 2 D }
  \int r \, f'(r) \, \partial_\rho y(r) \, d\vr,
\]
%
%
%
which can be compared directly to
the compressibility-route result
%
\[
\partial_\rho \left( \beta \, P^{(c)} \right)
=
1 - \rho \int c(\vr) \, d\vr.
\]



More generally,
we can compute derivatives with respect to,
instead of $\rho$,
the inverse temperature $\beta$,
or the charging parameter $\xi$
used in Section \ref{sec:kirkwood},
and use the corresponding self-consistency conditions
to determine additional parameters
in the integral equation.





\subsection{Application: the critical point of the Lennard-Jones fluid}





As an application,
we used the above method to locate the critical point
of the Lennard-Jones fluid\cite{
henderson1970, *henderson1972},
whose pair potential is
$\phi(r) = 4 (r^{-12} - r^{-6})$.
%
We used a simple closure with a single free parameter, $s$:
%
%
%
\begin{equation}
g(\vr)
=
\exp[-\beta\,\phi(\vr) + s\,\beta\,\phi_{\mathrm{lr}}(\vr)]
\, [1 + t(\vr) - s\,\beta\,\phi_{\mathrm{lr}}(\vr)],
\label{eq:msa_py}
\end{equation}
%
where
%
\begin{equation*}
\phi_{\mathrm{lr}}(r)
=
\begin{cases}
\phi(r) & \mbox{for $r > 2^{1/6}$}, \\
-1      & \mbox{for $r < 2^{1/6}$},
\end{cases}
\end{equation*}
%
Equation \eqref{eq:msa_py}
is an interpolation of
the soft-core mean spherical approximation
(SMSA, $s = 1$)\cite{chihara1973, *madden1980, *zerah1986}
and the PY one ($s = 0$).
%
We now use the self-consistency,
$\partial P^{(c)}/\partial \rho
= \partial P^{(v)}/\partial \rho$,
and the (compressibility-route) criticality condition,
\[
\partial P^{(c)} / \partial \rho
=
\partial^2 P^{(c)} / \partial \rho^2 = 0
\]
%
to determine $s$ and the critical density, $\rho_c$,
and the critical temperature, $T_c$.
%
Note that
all the derivatives can be computed from the above technique.
%
As shown in Table \ref{tab:ljcritical},
the critical point from Eq. \eqref{eq:msa_py}
lies in between the PY and SMSA ones,
and the overall agreement
with the simulation results\cite{perez-pellitero2006}
are slightly improved.



\begin{table}
\begin{tabular}{ l l l l }
\hline
&
$T_c$
&
$\rho_c$
\\ \hline
%
PY\cite{henderson1970, *henderson1972}
&
1.320
&
0.278
\\
%
SMSA
&
1.071
&
0.311
\\
%
Eq. \eqref{eq:msa_py}
&
1.300
&
0.297
\\ \hline
%
Simulation\cite{perez-pellitero2006}
&
1.3126(6)
&
0.3174(6)
\\ \hline
%
\end{tabular}
\caption{
Critical point of the Lennard-Jones fluid
(compressibility route).
}
\label{tab:ljcritical}
\end{table}





\section{\label{sec:cr_mayer}
Correlation functions from Mayer sampling}





Here we describe a method of computing the density components
of the correlation functions\cite{
perera2009, *dennison2009},
such as $c(\vr)$, $t(\vr)$, and $y(\vr)$,
by Mayer sampling\cite{
ree1966, rast1999, *labik2003, *kolafa2006, *kwak2005}.
%
The following version,
tailored for the hard-sphere fluid,
requires sampling of only biconnected diagrams,
and can thus be embedded in a regular Mayer sampling
in computing the virial coefficients\cite{zhang2014}.
%
This method is easy to implement
because the correlation functions
are computed from a few histogram-like quantities.




\subsection{\label{sec:cr_mayer_graphs}
Graphical expansion}




We first review a few key terms
in the graphical expansion
(using the convention of a previous study\cite{zhang2014}),
%
and the reader should consult
the relevant literature\cite{
uhlenbeck1962, hansen, ree1966, zhang2014,
morita1961III, *singh2004,
ree1964a, ree1964c, clisby2006, wheatley2013}
for a more comprehensive survey.
%
Virial coefficients and correlation functions
can be expressed as sums of \emph{diagrams}, or \emph{graphs},
each of which represents an integral.
%
In the diagram,
a regular vertex represents a spatial point $\vr_i$,
whose coordinates are to be integrated out,
and it contributes a factor of $\rho$ to the integrand.
%
A designated vertex,
or a \emph{root},
however, holds fixed and given coordinates.
%
An edge, or an \emph{$f$-bond},
joining two vertices $i$ and $j$
contributes a factor of
$f(\vr_{ij}) = \exp[-\beta \, \phi(\vr_{ij})] - 1$,
to the integrand.
%
A graph formed this way is called a \emph{Mayer diagram}.
%
A diagram in the expansion of a virial coefficient
has no root.
%
On the other hand,
a diagram in the expansion of a correlation function,
e.g., $c(\vr)$, has two roots,
with $\vr$ being the displacement between the spatial points
of the two roots.



We say that a graph is \emph{connected} if it is in one piece.
%
If removing a vertex $v$ in a connected graph
disconnects the graph,
$v$ is an \emph{articulation point}.
%
A connected graph with no articulation point
is \emph{biconnected}.
%
Similarly, if removing a pair of vertices $\{u, v\}$
breaks a connected graph,
$\{u, v\}$ is a separation pair.





\subsection{Ree-Hoover diagram}




We now introduce the \emph{Ree-Hoover (RH) diagram},
which joins every pair of nonadjacent vertices
in the Mayer diagram by an \emph{$e$-bond}.
%
The $e$-bond contributes a factor of
$\exp[-\beta\, \phi(\vr_{ij})] = f(\vr_{ij}) + 1$
to the integrand of the diagram.
%
The adjacency, (bi)connectivity, etc.,
of a RH diagram are defined in terms of the $f$-bonds.



Given an $n$-point configuration $\vr^n$,
we construct the \emph{intrinsic RH diagram},
$G(\vr^n)$, as follows.
%
For each vertex pair $\{i, j\}$,
we assign an $e$-bond between $i$ and $j$,
if $|\vr_{ij}| > 1$, or an $f$-bond, otherwise.
%
If $G(\vr^n)$ is biconnected,
then $\vr^n$ is a \emph{biconnected configuration}.



In the hard-sphere case,
it is readily shown that the integrand of an RH diagram
$G'$ is nonzero at $\vr^n$
if and only if $G'$ is identical to $G(\vr^n)$.
%
It follows that the domains of any two RH diagrams
do not overlap in the configuration space.
%
This result makes RH diagrams more useful than Mayer ones
in Monte Carlo (MC) calculations,
which aim at computing some sums of diagrams.
%
In the former case,
for each relevant configuration, $\vr^n$,
we only need to determine $G(\vr^n)$
and its weight in the sum,
whereas in the latter case,
we need to list all Mayer diagrams compatible with $\vr^n$.





\subsection{Virial coefficient}




For example,
the virial coefficient is equal to
a sum of biconnected Mayer diagrams.
%
The corresponding sum of RH diagrams can be written as
a single integral of all biconnected configurations:
%
%
%
\begin{equation}
B_n =
\frac{1 - n}{n!}
\int
f_B(\vr^n) \, \mathrm{bc}(\vr^n) \, d\vr^n
=
\frac{1-n}{n!}
Z_n \,
\langle f_B(\vr^n) \rangle_\mathrm{bc},
\label{eq:Bn_integral_fB}
\end{equation}
%
%
%
where $\mathrm{bc}(\vr^n)$ is 1.0
if $\vr^n$ is biconnected,
or zero otherwise,
and $Z_n \equiv \int \mathrm{bc}(\vr^n) \, d\vr^n$.
%
The $f_B(\vr^n)$,
called the \emph{signed star content},
summarizes the total contribution of $\vr^n$
to the sum\cite{uhlenbeck1962, hansen}.
%
In the second step,
the integral is converted to an average
over all biconnected configurations\cite{zhang2014}.
%
Note that,
in the hard-sphere case,
$f_B(\vr^n)$ depends on $\vr^n$
only through the intrinsic RH diagram, $G(\vr^n)$,
and $f_B(\vr^n)$ is a possibly-negative integer.





\subsection{Direct and indirect correlation functions}





Similar to Eq. \eqref{eq:Bn_integral_fB},
we can compute the density components $c_{n-2}(\vr)$
of the direct correlation function as
%
%
%
\begin{align}
c_{n-2}(\vr)
&=
\frac{1}{n!}
\sum_{i \ne j}
\int
  f_B(\vr^n) \, \delta(\vr_{ij} - \vr) \,
  \mathrm{bc}(\vr^n) \, d\vr^n
\notag \\
&=
\frac{Z_n}{n!}
\sum_{i \ne j}
\left\langle
  f_B(\vr^n) \,
  \delta(\vr_{ij} - \vr)
\right\rangle_{\mathrm{bc}},
\label{eq:cr_mayer}
\end{align}
%
%
%
where the factor $n!$ is due to the permutations of
the $n$ vertices.
%
One can readily verify
%
%
%
% Note, the $i \ne j$ sum gives $n (n - 1)$ equivalent pairs,
% So n! becomes (n-2)!.
%
%
%
$\int c_{n-2}(\vr) \, d\vr
= \langle f_B(\vr^n) \rangle_\mathrm{bc}
Z_n / (n - 2)! = -n \, B_n$.
%
%
%
Equation \eqref{eq:Bn_integral_fB}
shows that in a Mayer sampling over
all biconnected configurations,
$c_{n-2}(\vr)$ can be retrieved from
the histogram-like quantity
$\left\langle
  f_B(\vr^n) \, \delta(\vr_{ij} - \vr)
\right\rangle_\mathrm{bc}$.
%
The latter can be collected in a way
similar to that for the pair correlation function, $g(\vr)$,
in a regular MC or molecular dynamics simulation.
%
With $c_l(\vr)$ in hand,
we can compute the density components of
the indirect correction function, $t_l(\vr)$
from the OZ relation.





\subsection{Cavity distribution function}





We now turn to the density components of
the cavity distribution function, $y_l(\vr)$.
%
First, by definition, we have
%
%
%
\[
[1 + f(\vr)] \, y(\vr)
= g(\vr)
= 1 + t(\vr) + c(\vr),
\]
%
%
%
or its component form,
%
%
%
\begin{equation}
[1 + f(\vr)] \, y_l(\vr)
= \delta_{l0} + t_l(\vr) + c_l(\vr).
\label{eq:closure_component}
\end{equation}



For the hard-sphere fluid at $r > 1$,
we have $f(\vr) = 0$,
and $y_l(\vr) = c_l(\vr) + t_l(\vr)$,
which can thus be computed from the above technique.
%
The value of $y_l(\vr)$ inside the hard core ($r < 1$),
however, requires additional computation.



By comparing the diagrams with adjacent roots
in Eq. \eqref{eq:closure_component},
we find that the diagrams in the expansion of
$f(\vr) \, y_l(\vr)$ are precisely
the biconnected diagrams in $c_l(\vr)$
with adjacent roots.
%
By a comparison to Eq. \eqref{eq:Bn_integral_fB},
and by noticing $f(\vr) = -1$ for $r < 1$,
we find
%
%
%
\[
y_{n-2}(\vr)
=
-\frac{Z_n}{n!}
\sum_{i \ne j}
\left\langle
  f_B^{i,j}(\vr^n) \, \delta(\vr_{ij} - \vr)
\right\rangle_\mathrm{bc}
\quad
\mbox{for $r < 1$},
\]
%
%
%
where $f_B^{i,j}(\vr^n)$,
a modification of $f_B(\vr^n)$,
is the sum of integrands of all biconnected Mayer diagrams $H$
of $n$ vertices with two \emph{adjacent} roots $i$ and $j$:
%
%
%
\begin{equation}
f_B^{i,j}(\vr^n)
\equiv
\sum_{ \substack{
          \mathrm{bc}(H) \\
          \{i,j\}\in E(H) } }
\prod_{ \{k,l\} \in E(H) }
f_{kl},
\label{eq:fBij_def}
\end{equation}
%
%
%
where $E(H)$ is the set of edges in $H$.
%
Since the $f$-bond of the hard-sphere fluid
is either $0$ or $-1$,
the summand is $0$ or $\pm 1$.





\subsection{RH theorem}




To compute $f_B^{i,j}(\vr^n)$,
we need the following generalized
\emph{RH theorem}.
%
Let $R$ be a condition that applies to Mayer and RH diagrams.
%
We say that $R$ is \emph{compatible}
if it has the following property:
if $G$ satisfies $R$,
any graph $G'$ with additional $f$-bonds
also satisfies $R$.
%
For example,
the connectivity and biconnectivity
are compatible conditions.
%
We say that a diagram $G$ is \emph{$R$-restricted}
if $G$ satisfies some compatible condition $R$.
%
Now the generalized RH theorem states that
the uniform sum of all $R$-restricted Mayer diagrams
is equivalent to a weighted sum of
all $R$-restricted RH diagrams.
%
The weight, or the \emph{$R$-content},
of diagram $G$ in the RH sum is equal to
the number of $R$-restricted subgraphs of $G$
with even number of missing $f$-bonds,
less the number of $R$-restricted subgraphs of $G$
with odd number of missing $f$-bonds.



For a hard-sphere fluid,
since each configuration, $\vr^n$,
corresponds to only one diagram,
the intrinsic RH diagram, $G(\vr^n)$,
the overall contribution of $\vr^n$
to the uniform $R$-restricted Mayer sum
is given by the $R$-content of $G(\vr^n)$
multiplied by the integrand,
which is $(-1)^{N_f}$ for a graph of $N_f$ $f$-bonds.



For computing the virial coefficient,
the condition $R$ is the requirement for a graph $G$
to be biconnected (or to be a ``star'').
%
Then, the overall contribution $f_B(\vr^n)$ from $\vr^n$
to the uniform Mayer sum is equal to
the star content of $G(\vr^n)$ times $(-1)^{N_f}$.



Similarly,
to compute $f_B^{i,j}(\vr^n)$,
we set $R$ as the requirement for a graph $G$ to be biconnected
with adjacent roots $i$ and $j$.
%
Then,
$f_B^{i,j}(\vr^n)$
is equal to this particular $R$-content,
or the $y^{i,j}$-content,
of $G(\vr^n)$ times $(-1)^{N_f}$.
%
Algorithmically,
the $y^{i,j}$-content can be found
from a recursive search over all biconnected subgraphs $H$
of $G(\vr^n)$ with adjacent vertices $i$ and $j$
[note, $i$ and $j$ are joined before the search
if they are not adjacent in $G(\vr^n)$].
%
This gives the first and the direct method of
computing $f_B^{i,j}(\vr^n)$.





\subsection{Two variants of the cavity distribution function}





The above method can be extended to
two related correlation functions.
%
First,
consider $\omega_l(\vr)$,
the $l$th component of $\omega(\vr) \equiv \log y(\vr)$.
%
%Note that $\omega(\vr)$ is related to
%the bridge function as $d(\vr) \equiv \omega(\vr) - t(\vr)$.
%
In any diagram in the expansion of $\omega_l(\vr)$,
the roots cannot form a separation pair.
%
Therefore, to compute the corresponding weight,
$f_\omega^{i,j}(\vr^n)$,
instead of $f_B^{i,j}(\vr^n)$,
we should further exclude subgraphs in which $\{i,j\}$
forms a separation pair in the above search over
the biconnected subgraphs with adjacent roots.
%
Particularly,
if $\{i, j\}$ is a separation pair in $G(\vr^n)$,
then $f_\omega^{i,j}(\vr^n)$ is necessarily zero,
and the search is spared.



The second extension is to the function
\[
\Delta y_l
\equiv
y_l(\vr) - t_l(\vr)
= c_l(\vr) - f(\vr) \, y_l(\vr).
\]
The second expression, valid for $l \ge 1$,
shows that with all the biconnected graphs
with adjacent roots in $f(\vr) \, y_l(\vr)$
are removed from $c_l(\vr)$,
$\Delta y_l$ collects the biconnected graphs
with \emph{nonadjacent} roots.
%
In this case,
we set $R$ as the requirement for a graph $G$
to be biconnected with nonadjacent roots $i$ and $j$.
%
To compute the corresponding weight,
denoted as $f_B^{i,j*}(\vr^n)$,
we first remove the $f$-bond between $i$ and $j$,
if any, in $G(\vr)$ to form $G'$,
and then enumerate the biconnected subgraphs $H$ of $G'$,
and
%
%
%
\begin{align*}
\Delta y_{n-2}(\vr)
=
\frac{Z_n}{n!}
\sum_{i \ne j}
  \left\langle
    f_B^{i, j*}(\vr^n) \, \delta(\vr_{ij} - \vr)
  \right\rangle_\mathrm{bc}.
\end{align*}
%
%
%
Particularly,
if $G'$ is not biconnected,
then $f_B^{i,j*}(\vr^n) = 0$,
and no search is necessary.
%
For the hard-sphere fluid at $r > 1$,
the above algorithm is identical to that of
computing $f_B(\vr^n)$ for $c_l(\vr)$,
because in this case
$f(\vr) = 0$,
and
$\Delta y_l(\vr) = c_l(\vr) - f(\vr) \, y_l(\vr) = c_l(\vr)$.





\subsection{Wheatley's method}





Besides the above direct method,
we can also compute $f_B^{i,j}(\vr^n)$
by Wheatley's method\cite{wheatley2013}.
%
The main advantage is that
it groups diagrams
by the subsets of $V = \{1, \dots, n\}$
(the set of the $n$ vertices),
and computes only the collective sums of diagrams
for each subset,
instead of the value of each diagram individually.
%
Thus, it reduces the worst-case time complexity
from the order of $2^{n(n-3)/2}$ (direct method)
to $n\cdot3^n$.



We start by defining a modified Boltzmann weight,
$f_Q^{i,j}(S)$, for an arbitrary vertex subset $S$.
%
If $\{i, j\} \not\subset S$,
we use the usual definition
%
\[
f_Q^{i,j}(S)
\equiv
\prod_{\{k,l\}\subseteq S} (1 + f_{kl})
=
\sum_H
\prod_{\{k,l\} \in E(H)} f_{kl},
\]
%
where, in the first step,
the product is over all pairs $\{k, l\}$
of the vertices contained in $S$.
%
In the second step,
we have expanded the product
and represented each individual product of $f$-bonds
by a Mayer subgraph $H$ built on the vertex set $S$.
%
The outer sum is carried over all
(connected or disconnected) Mayer subgraphs $H$.
%
But, if $\{i, j\} \subset S$,
we define
%
%
%
\begin{equation}
f_Q^{i,j}(S)
\equiv
\prod_{ \substack{
  \{k,l\}\subseteq S \\
  \{k,l\} \ne \{i, j\} } }
(1 + f_{kl})
=
\sum_{ \substack{
  H \\
  \{i, j\} \in E(H) } }
\prod_{\{k,l\} \in E(H)} f_{kl},
\label{eq:fQ_ij}
\end{equation}
%
%
%
where the subgraph $H$
in the sum must have the $f$-bond $\{i, j\}$.



Now $f_B^{i,j}$ in Eq. \eqref{eq:fBij_def}
differs from $f_Q^{i,j}$ in Eq. \eqref{eq:fQ_ij}
by only the requirement of biconnectivity.
%
Thus,
to get $f_B^{i,j}$,
we only need to exclude the contribution of non-biconnected
diagrams from $f_Q^{i,j}$.
%
This is done by two recursions\cite{wheatley2013}.



The first recursion removes the contribution
of all disconnected diagrams,
to form a connected sum $f_C^{i,j}(S)$
for every subset $S$ of $V$\cite{wheatley2013}:
%
%
%
\begin{equation}
f_C^{i,j}(S)
=
f_Q^{i,j}(S)
-
\sum_{T \subset S}
f_C^{i,j}(T) \, f_Q^{i,j}(S \backslash T),
\label{eq:fC_from_fQ}
\end{equation}
%
%
%
where $S\backslash T$ denotes the set of vertices
that are in $S$, but not in $T$.
%
The sum is under two restrictions:
%
(i) $T$ must be a strict subset of $S$;
%
(ii) if both $i$ and $j$ belong to $S$,
they must belong to $T$ as well
(that is, the partition of $T$ and $S \backslash T$
should not separate $i$ and $j$).
%
Otherwise, $T$ must have the lowest vertex $v_0$ in $S$.
%
In this way,
a subset of $S$ occurs in either $T$ or $S \backslash T$
in the sum,
avoiding double counting.



The second recursion further excludes connected diagrams
with articulation points.
%
Let $f_{B,v}^{i,j}(S)$ be the sum of connected Mayer diagrams
with no articulation point lower than $v$,
then\cite{wheatley2013},
%
%
%
\begin{equation}
f_{B,v+1}^{i,j}(S)
=
f_{B,v}^{i,j}(S)
-
\sum_{\{v\} \subset T_v \subset S}
f_{B, v+1}^{i,j}(T_v)
\,
f_{B,v}^{i,j}(T_v^\dagger),
\label{eq:fB_from_fB}
\end{equation}
%
%
%
where
$T_v^\dagger = \left(S \backslash T_v\right) \cup \{v\}$.
%
The sum is under the following restriction.
%
If both $i$ and $j$ belong to $S$,
they must belong to $T_v$ too;
otherwise the lowest vertex $v_1$ in $S \backslash \{v\}$
must belong to $T_v$.
%
Initially,
we set $f_{B,1}^{i,j}(S) = f_C^{i,j}(S)$
from the previous step.
%
The final result is retrieved from
$f_B^{i,j}(V) = f_{B, n+1}^{i,j}(V)$.



For the hard-sphere fluid,
some graphical checks\cite{zhang2014}
can be used to save time.
%
For example,
the induced graph $G(S)$ must be connected for
$f_C^{i,j}(S)$ and $f_{B,v}^{i,j}(S)$
to be nonzero;
%
if $G(S)$ has an articulation point at $v$,
then $f_{B,u}^{i,j}(S) = 0$ for $u > v$.



Wheatley's method can also be used to compute
$\omega_l(\vr)$ and $\Delta y_l(\vr)$.
%
For $\omega_l(\vr)$,
we carry the same procedure for $y_l(\vr)$,
and then use the following recursion to
exclude diagrams that have $\{i, j\}$
as a separation pair:
%
\begin{align*}
f_\omega^{i,j}(S)
=
f_B^{i,j}(S)
+
\sum_{\{i,j\} \subset T \subset S}
f_\omega^{i,j}(T) \, f_B^{i,j}(T^\dagger),
\end{align*}
%
where both subsets $T$ and $T^\dagger$
contain $\{i, j\}$,
and the plus, instead of minus,
sign before the sum is due to that
both $f_B^{i,j}$ and $f_\omega^{i,j}$
contain an $f$-bond (equal to $-1$)
between $i$ and $j$.
%
In the end,
$f_\omega^{i,j}(V)$
can be used as $f_\omega^{i,j}(\vr^n)$
in the following equation:
%
%
%
\[
\omega_{n-2}(\vr)
=
\log y_{n-2}(\vr)
=
- \frac{Z_n}{n!}
\sum_{i\ne j}
  \left\langle
    f_\omega^{i,j}(\vr^n) \, \delta(\vr_{ij} - \vr)
  \right\rangle_\mathrm{bc}.
\]



To compute $\Delta y_l(\vr)$,
we should remove the factor $f_ij$
from the modified Boltzmann factor in Eq. \eqref{eq:fQ_ij}:
%
%
%
\begin{align*}
f_Q^{i,j*}(S)
\equiv
\prod_{\substack{
\{k,l\} \subseteq S \\
\{k,l\} \ne \{i, j\}
}}
  (1 + f_kl)
=
\sum_{\substack{
H' \\
\{i,j\} \notin E(H')
}}
\prod_{\{k,l\} \in E(H')}
f_{kl}.
\end{align*}
%
%
%
where the sum is carried over diagrams, $H'$,
with nonadjacent $i$ and $j$.
%
We then carry a regular Wheatley's recursion:
in Eqs. \eqref{eq:fC_from_fQ} and \eqref{eq:fB_from_fB},
we always require $T$ and $T_v$
to contain $\{v_0\}$ and $\{v_1\}$, respectively,
instead of $\{i, j\}$,
even if $\{i, j\} \subset S$.





\subsection{Computational techniques}




The direct and Wheatley's methods
are preferred for sparse and denser diagrams,
respectively.
%
In practice,
we use the former if the input diagram, $G$,
has no more than $2n-2$ $f$-bonds,
or the latter otherwise.



If the intrinsic RH diagram, $G(\vr^n)$,
has a clique separator
(a mutually-adjacent subset of vertices
whose removal disconnects the diagram)
that does not involve $\{i, j\}$,
then $f_B^{i,j}(\vr^n) = f_B^{i,j*}(\vr^n) = 0$.
%
The proof is similar to that for $f_B(\vr^n)$\cite{zhang2014}.
%
Since testing clique separators takes only polynomial time,
it generally saves time to do so
before using the direct or Wheatley's method
(both are exponential-time algorithms).



Finally, to avoid repetitive computation,
the look-up table techniques\cite{
zhang2014, ree1964a, clisby2006,
ree1967, *labik2005, *schultz2014},
developed for $f_B$,
can be adopted to accelerate the calculations.





\bibliography{liquid}





\end{document}

