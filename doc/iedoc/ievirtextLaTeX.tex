\documentclass[aip,jcp,preprint,superscriptaddress,showpacs,preprintnumbers,amsmath,amssymb]{revtex4-1}
%\documentclass{article}

\setcitestyle{super}

%\usepackage[a4paper, top=0.9in, bottom=1 in]{geometry}
\usepackage{graphicx}
\usepackage{amsmath}
\usepackage{amsthm}
\usepackage{mathrsfs}
\usepackage{url}
\usepackage{tikz}
\usepackage{wasysym}



%\numberwithin{subsection}{section}
%\numberwithin{subsubsection}{subsection}
%\renewcommand{\thesubsection}{\thesection.\Alph{subsection}}



\renewcommand{\thetable}{\Roman{table}}

\newcommand{\vct}[1]{\mathbf{#1}}
\providecommand{\vr}{} % clear \vr
\renewcommand{\vr}{\vct{r}}
\newcommand{\vk}{\vct{k}}
\newcommand{\vR}{\vct{R}}
\newcommand{\dvk}{\frac{d\vk}{(2\pi)^D}}
\newcommand{\tdvk}{\tfrac{d\vk}{(2\pi)^D}}





\begin{document}





\title{Computation of virial coefficients from integral equations}
\author{Cheng Zhang}
\author{Chun-Liang Lai}
\author{B. Montgomery Pettitt}
\email{mpettitt@utmb.edu}
\affiliation{
Sealy Center for Structure Biology and Molecular Biophysics,
Department of Biochemistry and Molecular Biology, \\
The University of Texas Medical Branch,
Galveston, TX 77555, USA}



\begin{abstract}
A polynomial-time method of
computing the virial coefficients
from an integral equation framework
is presented.
%
The method computes the truncated density expansions
of the correlation functions by series transformations,
and then extracts the virial coefficients
from the density components.
%
As an application,
the method was used in a hybrid-closure integral equation
with a set of self-consistent conditions,
which produced reasonably accurate virial coefficients
for the hard-sphere fluid and Gaussian model
in high dimensions.
\end{abstract}



\maketitle





\section{Introduction}

The virial expansion\cite{
mayer, hill, *rice1965, uhlenbeck1962, hansen, mcquarrie}
of a simple fluid
is a series for the pressure, $P$,
in terms of the number density, $\rho = N/V$:
%
\begin{equation}
\beta \, P
=
\sum_{n = 1}^{+\infty} B_n \, \rho^n
\label{eq:virser}
\end{equation}
%
where $\beta = 1/(k_B T)$
is the inverse temperature
($k_B$ is the Boltzmann constant),
and $B_n$ are the virial coefficients.
%
This expansion provides one of the simplest equation of states.
%
Further,
even beyond the radius of convergence of the series,
we can use the virial coefficients
to construct an equation of state
with a more suitable algebraic form
by the matching the low-density behavior\cite{wheatley2013}.



While the low-order ($n\le4$) virial coefficients
can be computed accurately\cite{
vanderwaals1899, *vanlaar1899,
*rushbrooke1961, *hemmer1965, *kim1968, *kilpatrick1971,
*dyer2001, *sun1996,
nijboer1952,
kratky1982,
luban1982, *joslin1982, *clisby2004a, *lyberg2005},
the computation of high-order ones remains challenging\cite{
wheatley2013, masters2008}.
%
This is so even for the simplest systems
such as the hard-sphere fluid\cite{
wheatley2013, kratky1982,
metropolis1953, *rosenbluth1954, *ree1964a,
*ree1967, *vanrensburg1993, *bishop1999,
*vlasov2002, *kolafa2004, *bishop2004, *bishop2005,
*labik2005, *clisby2005, *bishop2008,
clisby2004b, *clisby2006,
zhang2014, schultz2014},
%
in which spheres interact via
a repulsive hard-core potential
[that is,
the potential, $\phi(r)$, between two spheres of unit diameter
separated by a distance $r$ is zero if $r > 1$,
or infinity if $r < 1$].
%
Despite its simplicity,
the hard-sphere fluid is theoretically important,
and the system and its virial coefficients
have been intensively studied
not only in two and three dimensions,
but also in much higher dimensions\cite{
luban1982, *joslin1982, *clisby2004a, *lyberg2005,
clisby2004b, *clisby2006,
adda-bedia2008a, *adda-bedia2008b,
*finken2001, *frisch1999, *frisch1985, *kurchan2012,
*loeser1991, *parisi2000, *robles2004, *robles2007,
*rohrmann2007, *rohrmann2008, *skoge2006,
*torquato2006a, *torquato2006b,
*wyler1987}.
%
One also hopes that
the insight gained from high dimensions
can be useful for two and three dimensions.



In higher dimensions,
reasonably accurate high-order virial coefficients
for the hard-sphere fluid can be obtained\cite{zhang2014}
by Mayer sampling\cite{schultz2014,
rast1999, *labik2003, *singh2004, *naresh2009a, *naresh2009b,
*benjamin2007, *schultz2009, *shaul2010, *tan2010,
*shaul2012, *schultz2013, *yang2013,
kwak2005, shaul2011}.
%
The resulting virial coefficients
showed a regular pattern as the order $n$ changes.
%
Thus, we wish to consider
whether they can be derived,
at least approximately,
by some recurrence relation.



One way to answer the question is
through the correlation functions,
which can be approximately determined
by integral equations,
such as the Percus-Yevick (PY)\cite{percusyevick}
or hypernetted-chain (HNC)\cite{
morita1958, *morita1959, morita1960,
vanleeuwen1959, *meeron1960, *verlet1960, *green1960, *rushbrooke1960}.
%
Using such integral equations,
one can readily compute
the density expansions of the correlation functions,
and then extract the virial coefficients
from the integrals of the density components\cite{
shaul2011}.
%
Although previous algorithms\cite{shaul2011}
work well for the PY equation,
the HNC version is somewhat more intensive computationally,
for high orders
because of its exponential time complexity.
%
Further,
since neither the PY nor the HNC integral equation
produces reliable virial coefficients for high orders
or in higher dimensions, we wish to consider alternative routes.



Here,
we generalize the algorithm of Shaul et al.\cite{shaul2011}
to compute the virial coefficients
from a larger class of integral equations.
%
Our method has a polynomial time complexity,
and thus computationally is particularly useful
for high orders.
%
As an application,
we used the algorithm in a hybrid integral equation
with a set of order-by-order self-consistent conditions.
%
The method yields rather accurate virial coefficients
of the high-dimensional hard-sphere fluid and Gaussian model.
%
We describe the method in Sec. II,
use it on the hard-sphere fluid and Gaussian model in Sec. III,
and conclude the article in Sec. IV.




\section{Method}





\subsection{Integral equations}




In liquid sate theory,
the integral equation method\cite{hansen}
is a way of computing the pair correlation function, $g(\vr)$.
%
The equation can be cast as a pair of coupled equations
involving the direct correlation function, $c(\vr)$,
and the indirect correlation function,
$t(\vr) \equiv g(\vr) - c(\vr) - 1$.
%
For an isotropic fluid, as assumed here,
the above correlation functions
depends on the coordinates, $\vr$,
only through the distance, $r = |\vr|$.



The first of the coupled equations
is the Ornstein-Zernike (OZ) relation:
%
%
%
\begin{equation}
t(\vr)
=
\rho \, (c*h)(\vr)
= \rho \, [c*(c+t)](\vr),
\label{eq:oz}
\end{equation}
%
%
%
where ``$*$'' denotes a spatial convolution
$(a*b)(\vr) = \int a(\vr - \vr') \, b(\vr') \, d\vr'$,
and
$h(\vr) \equiv c(\vr) + t(\vr) = g(\vr) - 1$
is the total correlation function.
%
This relation is most convenient in Fourier space,
as it assumes the product form,
%
%
%
\begin{equation}
\tilde t(\vk)
=
\rho \, \tilde c(\vk)
\, [ \tilde c(\vk) + \tilde t(\vk) ],
\label{eq:oz_k}
\end{equation}
%
%
%
where
$\tilde a(\vk) = \int a(\vr) \, e^{-i\vk \cdot \vr} \, d\vr$
denotes the Fourier transform.
%
In Supplemental Material (SM) 1.A,
we describe two techniques
of computing the Fourier transforms
for spherical-symmetric functions
in odd and even dimensions.



The second equation, or the \emph{closure},
can be written in terms of the cavity distribution function,
which is defined as $y(\vr) \equiv g(\vr) \, \exp[\beta \, \phi(\vr)]$,
and
%
%
%
\begin{equation}
c(\vr)
=
[1 + f(\vr)] \, y(\vr) - 1 - t(\vr),
\label{eq:closure_cr_yr}
\end{equation}
%
%
%
where $f(\vr) \equiv \exp[-\beta \, \phi(\vr)] - 1$.
%
The formally-exact expression of $y(\vr)$ is
%
%
%
\[
y(\vr)
=
\exp[
t(\vr) + B(\vr)
],
\]
%
%
%
where $B(\vr)$
is the bridge function,
which cannot be written as a simple convolution expression
in terms of pair functions discussed above\cite{hansen}.
%
A closure without $B(\vr)$ most often
approximates $y(\vr)$ as a simple functional of $t(\vr)$.
%
The PY equation uses the following closure\cite{percusyevick}:
%
%
%
\begin{equation}
y(\vr) \approx 1 + t(\vr).
\label{eq:py}
\end{equation}
%
%
%
Alternatively,
the HNC equation assumes\cite{
morita1958, *morita1959, morita1960,
vanleeuwen1959,
*meeron1960, *verlet1960, *green1960,
*rushbrooke1960}
%
%
%
\begin{equation}
y(\vr)
\approx
\exp t(\vr).
\label{eq:hnc}
\end{equation}
%
%
%
In either case,
the closure along with the OZ relation, \eqref{eq:oz},
fully determines the approximate correlation functions,
$c(\vr)$ and $t(\vr)$.
%
Below we will illustrate our method
with the PY and HNC equations,
and the extension to more advanced integral equations\cite{
hiroike1957, *lee1995,
hurst1965,
rowlinson1965,
rowlinson1966,
lado1967, *henderson1970, *zerah1986,
caligaris1968, *zhou1990, *attard1991, *henderson1996,
gaskell1968,
stell1969, *rosenfeld1979,
hutchinson1971,
hutchinson1972,
verlet1980,
hall1980,
martynov1983,
rogers1984,
ballone1986,
duh1995, *bomont2001, *choudhury2002,
marucho2007}
is straightforward.
%
The latter equations often improve
the accuracy by interpolating the PY and HNC ones.





\subsection{\label{sec:PY}
PY equation}





We now compute the virial coefficients
from the PY equation\cite{percusyevick}.
%
First,
we expand the direct correlation function
as series of the density, $\rho$, as
%
\begin{equation}
c(\vr)
=
c_0(\vr)
+ \rho \, c_1(\vr)
+ \rho^2 \, c_2(\vr)
+ \cdots
\label{eq:crser}
\end{equation}
%
and the indirect correlation function as
%
\begin{equation}
t(\vr)
=
\rho \, t_1(\vr)
+ \rho^2 \, t_2(\vr)
+ \cdots
\label{eq:trser}
\end{equation}
%
where $c_0(\vr) = f(\vr)$,
and $t_0(\vr)$ vanishes.
%
The components of $c_l(\vr)$ and $t_l(\vr)$
are obtained below.



Plugging Eqs. \eqref{eq:crser} and \eqref{eq:trser}
into \eqref{eq:oz_k},
and comparing the coefficients for $\rho^l$ yields,
for $l \ge 1$,
\begin{equation}
\tilde t_l(\vk)
=
\sum_{j = 0}^{l - 1}
\tilde c_{l - 1 -j}(\vk)
\,
[\tilde c_j(\vk) + \tilde t_j(\vk)].
\label{eq:oz_tl}
\end{equation}



Similarly, for the closure, Eq. \eqref{eq:closure_cr_yr},
we have
\begin{equation}
c_l(\vr)
=
[1 + f(\vr)] \, y_l(\vr)
- t_l(\vr) - \delta_{l0}.
\label{eq:closure_cl_yl}
\end{equation}
%
In the PY case,
Eq. \eqref{eq:py}
means $y_l(\vr) = \delta_{l0} + t_l(\vr)$,
and thus,
\begin{equation}
c_l(\vr)
= f(\vr) \, \delta_{l0} + f(\vr) \, t_l(\vr).
\label{eq:py_cl}
\end{equation}
%
Equations \eqref{eq:oz_tl} and \eqref{eq:py_cl}
can be applied in succession:
%
%
%
\begin{align*}
&
c_{l-1}(\vr)
\xrightarrow{ \mathrm{F.T.} }
\tilde c_{l-1}(\vk)
\xrightarrow{ \mathrm{Eq.\,\eqref{eq:oz_tl}} }
\\
&
\tilde t_l(\vk)
\xrightarrow{ \mathrm{F.T.} }
t_l(\vr)
\xrightarrow{ \mathrm{Eq.\,\eqref{eq:py_cl}} }
c_l(\vr).
\end{align*}
%
%
%
to get higher-order density components of the correlation functions.
%
for example,
in the first iteration,
Eq. \eqref{eq:oz_tl} gives
$\tilde t_1(\vk) = [\tilde c_0(\vk)]^2 = [\tilde f(\vk)]^2$,
i.e.,
$t_1(\vr) = (f*f)(\vr)$;
then, Eq. \eqref{eq:py_cl} gives
$c_1(\vr) = (f*f)(\vr) \cdot f(\vr)$.
%
Note that the correlation functions
on the right-hand side \eqref{eq:oz_tl}
have orders lower than $l$;
%
thus those orders are always available
when we compute $\tilde t_l(\vk)$.





\subsubsection{Virial coefficients}





We can compute the virial coefficients
from the correlation functions, $c_l(\vr)$ and $t_l(\vr)$,
in a few ways.
%
First, in the compressibility route,
%
\begin{equation}
\partial_\rho( \beta \, P^{(c)} )
=
1 - \rho \int c(\vr) \, d\vr,
\label{eq:P_compressibility}
\end{equation}
%
where $\partial_\rho \equiv \partial/\partial \rho$.
%
Then, expanding both sides as series of $\rho$
using Eqs. \eqref{eq:virser} and \eqref{eq:crser} yields
%
\begin{equation}
B_n^{(c)}
=
-\frac{1}{n}
\int c_{n-2}(\vr) \, d\vr.
\label{eq:Bn_compressibility}
\end{equation}



Next, in the virial route,
we have, in $D$ dimensions,
%
%
%
\begin{align}
\beta \, P^{(v)}
&=
\rho
-
\frac{ \rho^2 \, \beta }
     {  2 \, D }
\int (\vr \cdot \nabla \phi) \, g(\vr) \, d\vr
\notag \\
&=
\rho
+
\frac{ \rho^2 }
     { 2 \, D }
\int (\vr \cdot \nabla f) \, y(\vr) \, d\vr.
\label{eq:P_virial}
\end{align}
%
%
%
So, for an isotropic fluid,
we have, for $n \ge 2$,
%
%
%
\begin{equation}
B_n^{(v)}
=
\frac{1}{2D}
\int r \, f'(r) \, y_{n-2}(r) \, d\vr.
\label{eq:Bn_virial}
\end{equation}
%
%
%
For the hard-sphere fluid,
$f'(r) = \delta(r - 1)$,
which is the $\delta$-function peaked at $r = 1$,
and Eq. \eqref{eq:Bn_virial} is reduced to
%
%
%
\[
B_n^{(v)} = B_2 \, y_{n-2}(1),
\]
%
%
%
where
$B_2 =
\left.
  \int r \, f'(r) \cdot 1 \cdot d\vr
\middle/
  (2 \, D)
\right.
=
\left.
  \pi^{D/2}
\middle/
  [ D \, \Gamma(D/2) ]
\right.$
is half of the volume of the unit sphere in $D$ dimensions,
and $\Gamma(z) = (z-1)!$ is the gamma function.
%
Since the closure is approximate,
Eqs. \eqref{eq:Bn_compressibility} and \eqref{eq:Bn_virial}
generally yield different values for $n \ge 4$.



In addition to
Eqs. \eqref{eq:Bn_compressibility} and \eqref{eq:Bn_virial},
which apply to a general integral equation,
where are two formulae specific to the PY equation.
%
Baxter's pressure formula\cite{baxterpressure}
can be written as
%
%
%
\begin{align}
\beta \, P^{(c)}
&=
\int \log[ 1 - \rho \, \tilde c(\vk) ] \, \dvk
-\frac{ \rho \, t(\vct{0}) + \rho^2 \, \tilde c(\vct 0) } { 2 }
\notag \\
&=
\rho
-
\rho^2
\int
\left\{
  c(\vr)
  + \frac{ [ 1 + t(\vr)]^2 \, (\vr \cdot \nabla f) } { 2 \, D }
\right\}
\, d\vr,
\label{eq:Pc_Baxter}
\end{align}
%
%
%
where we have used the virial theorem\cite{
rushbrooke1953, uhlenbeck1962}
in the second step (see SM 2).
%
The second expression of Eq. \eqref{eq:Pc_Baxter}
is valid even if $\rho \, \tilde c(\vk)$ exceeds 1.0.
%
For the hard-sphere fluid,
it can further simplified as
%
\begin{align*}
-\beta \, P^{\mathrm{ex},(c)}
=
\rho^2 \, [ \tilde c(0) + c^2(1^-) \, B_2]
=
\rho^2 \, [\tilde c(0) + g^2(1^+) \, B_2],
\end{align*}
%
where the ``ex'' on the superscript means the excess (non-ideal-gas) part.
%
Taking the $n$th component of the density expansion yields
a different expression for $B_n$.
%
The resulting value, however, is the same as $B_n^{(c)}$\cite{baxterpressure}.



The second derivative of pressure can be written as (see SM 2)
%
\[
-\beta \, \chi^\mathrm{ex}
\equiv
-\rho^2 \, \partial_\rho^2 (\beta \, P^\mathrm{ex})
\approx
\rho^2 \int [c(\vr) + h(\vr) \, t(\vr)] \, d\vr.
\]
%
Thus,
%
%
%
\begin{equation}
B_n^{(\mathrm{PY}, \chi)}
\approx
\frac{1}{ n \, (1 - n) }
\int
  \left[
    c_{n-2}(\vr)
    +
    \sum_{j = 1}^{n-2} h_{n-2-j}(\vr) \, t_j(\vr)
  \right]
  \, d\vr.
\label{eq:Bn_ddP}
\end{equation}
%
%
%
This value, referred to as the $\chi$-route virial coefficient below,
differs from $B^{(c)}$ and $B^{(v)}$.
%
Since the above virial coefficients are computed
from numerical integration,
their precision can be improved
by an extrapolation on the grid size (see SM 1.B).





\subsection{\label{sec:HNC}
HNC equation}




We now turn to the HNC case.
%
The OZ relation, Eq. \eqref{eq:oz_tl}, is still applicable.
%
Below,
we determine the density expansion of
the HNC cavity distribution function
$y(\vr) = \exp t(\vr)$:
%
%
%
\begin{equation}
y(\vr)
=
\sum_{l = 0}^{+\infty}
y_l(\vr) \, \rho^l,
\label{eq:yrser}
\end{equation}
%
%
%
with $y_0(\vr) = 1$,
such that the components $y_l(\vr)$
can be used in Eq. \eqref{eq:closure_cl_yl}.
%
Previously,
Shaul et al. proposed the following formula\cite{shaul2011}
%
\[
y_l(\vr)
=
\sum_{k = 0}^l
  \frac{1}{k!}
  \sum_{ \substack{
    (l_1, \dots, l_k) \\
    l_1 + \cdots + l_k = l } }
    t_{l_1}(\vr) \cdots t_{l_k}(\vr),
\]
%
where the inner sum is carried over
all possible $k$-tuples of positive integers
$(l_1, \dots, l_k)$.
%
The formula, however,
becomes difficult to use for a high order $l$
because of the exponentially growing number of the $k$-tuples.
%
Below we propose an algorithm with polynomial scaling of time.



First, differentiating Eq. \eqref{eq:hnc} with respect to $\rho$ yields
%
\[
\partial_\rho y(\vr)
=
y(\vr) \, [ \partial_\rho t(\vr) ].
\]
%
Then, expanding both sides
using Eqs. \eqref{eq:trser} and \eqref{eq:yrser}
gives, for $l\ge 1$,
%
\begin{equation}
y_l(\vr)
=
\sum_{j=1}^l
y_{l-j}(\vr) \, j \, t_j(\vr) / l.
\label{eq:hnc_yl}
\end{equation}
%
In Eq. \eqref{eq:hnc_yl},
the $y_{l-j}(\vr)$ on the right-hand side
have orders lower than $l$.
%
So we can readily compute $y_l(\vr)$
using an increasing order of $l$
[with $y_0(\vr) = 1$].
%
For the first few orders,
we have [dropping the ``$(\vr)$''],
%
$y_1 = y_0 \, t_1 = t_1$,
%
$y_2
= (y_1 \, t_1 + 2 \, y_0 \, t_2) / 2
= \frac{1}{2} t_1^2 + t_2$,
and
$y_3
= (y_2 \, t_1 + 2 \, y_1 \, t_2 + 3 \, y_0 \, t_3)/3
= \frac{1}{6} t_1^3 + t_1 \, t_2 + t_3$,
\dots, etc.



Once $y_l(\vr)$ is known,
we can compute $c_l(\vr)$ from Eq. \eqref{eq:closure_cl_yl}.
%
Applying Eqs.
\eqref{eq:oz_tl},
\eqref{eq:hnc_yl}, and
\eqref{eq:closure_cl_yl} in succession as
%
\begin{align*}
&
c_{l-1}(\vr)
\xrightarrow{ \mathrm{F.T.} }
\tilde c_{l-1}(\vk)
\xrightarrow{ \mathrm{Eq.}\,\eqref{eq:oz_tl} }
\tilde t_l(\vk)
\xrightarrow{ \mathrm{F.T.} }
\\
&
t_l(\vr)
\xrightarrow{ \mathrm{Eq.}\,\eqref{eq:hnc_yl} }
y_l(\vr)
\xrightarrow{ \mathrm{Eq.}\,\eqref{eq:closure_cl_yl} }
c_l(\vr),
\end{align*}
yields the correlation functions of any order $l$ in polynomial time.



We can generalize the above technique
to a larger class of closures
$y(\vr) = Y[t(\vr)]$\cite{
hurst1965, rowlinson1965, rowlinson1966,
lado1967, *henderson1970, *zerah1986,
hutchinson1971, hutchinson1972, verlet1980, hall1980,
martynov1983, rogers1984, ballone1986,
marucho2007} (SM 1.C),
and to the Yvon-Born-Green (YBG)\cite{
yvon1935, *born1946, hill, *rice1965, hansen, nijboer1952}
and Kirkwood\cite{
kirkwood1935, *stell1962, hoover1962, hill, *rice1965}
integral equations
(see SM 1.D and 1.E, respectively).





\subsubsection{Virial coefficients}





The compressibility- and virial-route virial coefficients
$B^{(c)}$ and $B^{(v)}$,
can be still computed from
Eqs. \eqref{eq:Bn_compressibility}
and
\eqref{eq:Bn_virial}, respectively.
%
Additionally,
the virial coefficient
can be derived from HNC-specific expressions
of the chemical potential $\mu$
and the free energy\cite{
morita1958, *morita1959, morita1960,
morita1960I, morita1961III,
singer1985}
(see also SM2).
%
For example,
using the chemical potential formula with
$\partial_\rho P = \rho \, \partial_\rho \mu$ yields
%
\[
B_n^{(\mathrm{HNC}, \mu)}
=
\frac{1-n}{n}
\int
\left[
  c_{n-2}(\vr)
  -\frac{1}{2} \sum_{j=1}^{n-2} h_{n-2-j}(\vr) \, t_j(\vr)
\right] \, d\vr.
\]
%
The value, however,
is the same as $B_n^{(v)}$\cite{
morita1960, morita1960I}
because of the virial theorem\cite{rushbrooke1953, uhlenbeck1962}
(see SM 2 for a proof).
%
In terms of the graphical expansion\cite{
hansen, uhlenbeck1962, morita1961III},
the integrands of
$B_n^{\mathrm{HNC}, \mu}$
and
$B_n^{(v)}$
differ only by the choices of the root vertices.



For the hard-sphere fluid,
we have an additional cavity-route virial coefficient\cite{hoover1962}
(see also SM 2):
%
\begin{equation}
B_n^{(y)}
=
\frac{n - 1}{n}
\left[
  \log y(\vct 0)
\right]_{n - 1}
\label{eq:Bn_cavity}
\end{equation}
%
where
$[ \log y(\vct 0) ]_{n - 1}$,
the coefficient of $\rho^{n-1}$
in the density expansion of $\log y(\vct 0)$,
can be computed from the components,
$y_l(\vct 0)$ ($l = 0, \dots, n - 1$).
%
This relation,
however,
demands a highly accurate $y(\vct 0)$,
which makes it applicable to very few cases,
such as the HNC equation in high dimensions.





\subsection{\label{sec:selfconsistent}
Detailed self-consistent (DSC) equation}





Neither the PY nor the HNC equation yields
the same virial coefficients from
the compressibility and virial routes.
%
Here we consider an interpolation between the PY and HNC equations
to force the consistency\cite{
hurst1965, rowlinson1965, rowlinson1966,
lado1967, *henderson1970, *zerah1986,
stell1969,
rogers1984,
rosenfeld1979,
hutchinson1971, hutchinson1972,
hall1980}
of all virial coefficients.
%
We assume a form for the new cavity distribution function to be
%
\begin{align*}
y(\vr)
&=
y^{\mathrm{PY}}(\vr)
+ \mathrm{correction}
\\
&=
1 + t(\vr)
+ \sum_{l = 2}^{+\infty} \lambda_l \, w_l(\vr) \, \rho^l,
\end{align*}
%
where
$w_l(\vr)$ is the $l$th order correction function,
and the parameter $\lambda_l$
is to be determined.
%
Particularly, we set
%
\begin{equation}
w_l(\vr)
=
y_l^{(\mathrm{HNC})}(\vr)
-
y_l^{(\mathrm{PY})}(\vr)
=
\left[ \exp t(\vr) \right]_l - t_l(\vr),
\label{eq:yr_trial}
\end{equation}
%
with
$[ \exp t(\vr) ]_l$
denoting the coefficient of $\rho^l$
in the density expansion of
$\exp t(\vr)$.
%
Thus,
$w_2(\vr)
= [t_1^2(\vr)]^2/2
= [(f*f)(\vr)]^2/2$,
and
$w_3(\vr)
= t_1(\vr) \, t_2(\vr)
+ [t_1(\vr)]^3/6$.
%
Below we determine the parameter $\lambda_l$
to force the equality of
Eqs. \eqref{eq:Bn_compressibility} and \eqref{eq:Bn_virial}
for each order $l$.



From Eq. \eqref{eq:closure_cr_yr}, we have
\begin{align*}
c(\vr)
=
f(\vr) \, [1 + t(\vr)]
+
[1 + f(\vr)]
\sum_{l = 2}^{+\infty}
  \lambda_l \, w_l(\vr) \, \rho^l.
\end{align*}
%
So
%
\begin{align}
c_l(\vr)
= f(\vr) \, \delta_{l0} + f(\vr) \, t_l(\vr)
+ \lambda_l \, [ 1 + f(\vr) ] \, w_l(\vr).
\label{eq:cl_corrected}
\end{align}
%
Note that the for the hard-sphere potential,
$1+f(\vr)$ is zero inside the core,
and thus the correction $w_1(\vr)$ exists only outside the core.



Now, by demanding
%
\begin{equation}
B_n^{(c)} = B_n^{(v)},
\label{eq:Bc_eq_Bv}
\end{equation}
%
and using Eqs. \eqref{eq:Bn_compressibility},
\eqref{eq:Bn_virial},
and
\eqref{eq:cl_corrected},
%
we get
%
\begin{equation}
\lambda_{n-2}
=
-
\frac{
  B_n^{(c, \mathrm{uncorr})}
- B_n^{(v, \mathrm{uncorr})}
} {
  \Delta B_n^{(c)}
- \Delta B_n^{(v)}
},
\label{eq:lambda}
\end{equation}
%
where
%
\begin{align*}
B_n^{(c, \mathrm{uncorr})}
\equiv
-\frac{1}{n}
\int f(\vr) \, t_{n-2}(\vr) \, d\vr,
\end{align*}
%
\begin{align*}
B_n^{(c, \mathrm{uncorr})}
\equiv
\frac{1}{2 \, D}
\int \vr \cdot \nabla f(\vr) \, t_{n-2}(\vr) \, d\vr,
\end{align*}
%
\begin{align*}
\Delta B_n^{(c)}
\equiv
-\frac{1}{n}
\int [f(\vr) + 1] \, w_{n-2}(\vr) \, d\vr,
\end{align*}
%
and
%
\begin{align*}
\Delta B_n^{(v)}
\equiv
\frac{1}{2 \, D}
\int \vr \cdot \nabla f(\vr) \, w_{n-2}(\vr) \, d\vr,
\end{align*}
%
The cases of $\lambda_l \equiv 0$ and $\lambda_l \equiv 1$
correspond to the PY and HNC closures, respectively.
%
A large deviation from $0$ and $1$ can mean an unstable solution.



Note that Eq. \eqref{eq:Bc_eq_Bv}
differs from the commonly-used self-consistent condition\cite{
hurst1965, rowlinson1965, rowlinson1966,
lado1967, *henderson1970, *zerah1986,
hutchinson1971, hutchinson1972,
verlet1980,
hall1980,
martynov1983,
rogers1984,
ballone1986,
duh1995, *bomont2001, *choudhury2002,
marucho2007},
which demands the same compressibility or pressure from
Eqs. \eqref{eq:P_compressibility} or \eqref{eq:P_virial}
at a certain density,
in that the former is a set of order-by-order conditions
instead of a single condition.
%
For this reason,
we refer to Eq. \eqref{eq:Bc_eq_Bv}
as the detailed self-consistent (DSC) condition.
%
Some generalizations are discussed in SM 1.F and 1.G.




\subsubsection{Empirical correction}





Ideally, the DSC condition, \eqref{eq:Bc_eq_Bv},
would yield the exact result if the functional form
of $w_l(\vr)$ were correct.
%
Since this is generally not so,
we give an empirical first-order correction below.
%
If we assume that
the $w_l(\vr)$ given by Eq. \eqref{eq:yr_trial}
deviates from the correct functional form
by a small amount $\epsilon$,
then the virial coefficients
from the two routes should also differ
by an amount proportional to $\epsilon$,
to the leading order.



Instead of a strict self-consistency, we now set
\begin{equation}
B_n^{(c)}
=
(1 + \kappa_n) \, B_n^{(v)},
\label{eq:Bc_eq_Bv_kappa1}
\end{equation}
with the parameters $\kappa_n$ of order $\epsilon$.
%
We recover Eq. \eqref{eq:Bc_eq_Bv}
if $\kappa_n = 0$.
%
Equation \eqref{eq:lambda} becomes
\begin{equation}
\lambda_{n - 2}
=
-
\frac{ B_n^{(c, \mathrm{uncorr})}
 - (1 + \kappa_n) \, B_n^{(v, \mathrm{uncorr})} }
{ \Delta B_n^{(c)}
  - (1 + \kappa_n) \, \Delta B_n^{(v)} }.
\end{equation}
%
Now that $B_n^{(c)} \ne B_n^{(v)}$,
we shall use $B_n^{(c)}$
as the best estimate
(which is usually more accurate than $B_n^{(v)}$,
as shown in Sec. \ref{sec:cmp_closures}):
%
\begin{equation}
B_n^{(c)}
=
B_n^{(c, \mathrm{uncorr})}
+
\lambda_{n - 2} \, \Delta B_n^{(c)}.
\end{equation}
%
For simplicity,
we will focus on a constant $\kappa_n$
that is turned on for $n \ge n_0$.
%
Once the $\kappa_n$
and $n_0$ are tuned for low-order virial coefficients,
e.g., by matching the known values,
we anticipate them to also improve
higher-order virial coefficients.





\subsubsection{
Density-dependent parameter and $\lambda$-DSC equation
}




In contrast to Eq. \eqref{eq:yr_trial},
we can alternatively set the parameter $\lambda$
as a power series of the density, $\rho$\cite{
rowlinson1965, gaskell1968, hutchinson1971, hall1980}:
\begin{equation}
\lambda
=
\lambda_2
+ \lambda_3 \, \rho
+ \lambda_4 \, \rho^2 + \cdots.
\label{eq:DSC_lambda}
\end{equation}
%
The coefficients $\lambda_l$
are to be determined to satisfy Eq. \eqref{eq:Bc_eq_Bv}
following a procedure in SM 1.F.
%
We refer to this variant as the $\lambda$-DSC equation below,
to distinguish it from the integral equation
based on Eq. \eqref{eq:yr_trial},
which is simply referred to as the DSC equation.
%
The $\lambda$-DSC equation is superior for low orders
in low dimensions,
whereas the DSC equation based on Eq. \eqref{eq:yr_trial}
is preferred for higher orders or in higher dimensions (SM 1.F).




\section{\label{sec:results}
Results}




In Sec. \ref{sec:cmp_closures},
the method described in Secs. \ref{sec:PY} and \ref{sec:HNC}
was used to compute the virial coefficients from
many existing integral equations.
%
We further show in Sec. \ref{sec:hs_results}
and \ref{sec:gaussian_results}
that the DSC equation described in \ref{sec:selfconsistent}
yielded reasonably accurate virial coefficients
for the hard-sphere fluid and Gaussian model in high dimensions,
although it is less successful in low dimensions.




\subsection{\label{sec:cmp_closures}
Comparison of integral equations}






To show the generality of the method,
we computed the virial coefficients from
a few integral equations based on the OZ relation, Eq. \eqref{eq:oz},
as listed in Table \ref{tab:closures},
as well as those from the YBG\cite{
yvon1935, *born1946, nijboer1952, hill, *rice1965, hansen}
and Kirkwood\cite{kirkwood1935, hill, *rice1965, hoover1962, stell1962}
equations, using the techniques in SM 1.C, 1.D and 1.E.



For an integral equation with a free parameter,
one usually adjusts the parameter to achieve
the consistency of
Eqs. \eqref{eq:P_compressibility} and \eqref{eq:P_virial}
at a certain density.
%
Doing so for the low density limit
yields the same $B_4$
(also $B_5$ for the Verlet equation\cite{verlet1980})
from the compressibility and virial routes.
%
The resulting parameters, as listed in Table \ref{tab:virhs3d},
may differ slightly from the literature values
chosen for the consistency at another density.
%
Note that the above consistency condition,
referred to as the $B_4$-SC condition below,
is limited to the lowest order only,
and thus higher-order virial coefficients from the two routes
would still differ.



The virial coefficients of the integral equations
are shown in Table \ref{tab:virhs3d}
for the three-dimensional hard-sphere fluid
after the above parameter adjustment.
%
The low-order results agreed with the literature values.
%
Except for the YBG, Kirkwood, and HNC cases,
most equations produced reasonably accurate
compressibility-route virial coefficients, $B^{(c)}$,
up to about the tenth order,
although $B_{11}$ and $B_{12}$ were less satisfactory.
%
This could explain the previous success of
some of the integral equations
over a wide density range.
%
For higher orders,
$B^{(c)}$ were more accurate than
the virial-route results, $B^{(v)}$
[this prompted us to use the former
in the modified DSC equation,
Eq. \eqref{eq:Bc_eq_Bv_kappa1}.
%
The cavity-route results, $B^{(y)}$,
however, were usually much less accurate than
$B^{(c)}$ and $B^{(v)}$.



\begin{table}
\begin{tabular}{ l l l l }
\hline
Closure
&
$y(\vr) = Y[t(\vr)]$
\\ \hline
%
PY\cite{percusyevick}
&
$y = 1+t$.
\\
HNC\cite{
morita1958, *morita1959, morita1960,
vanleeuwen1959,
*meeron1960, *verlet1960, *green1960,
*rushbrooke1960}
&
$y = \exp t$.
\\
Hurst\cite{hurst1965}
&
$t = y^m \, \log y$.
\\
Rowlinson 1\cite{rowlinson1965}
&
$t = \eta \, \log y + (1 - \eta) \, (y - 1)$.
\\
Rowlinson 2\cite{rowlinson1966}
&
$y = (1 - \eta) \, (1 + t) + \eta \, \exp t$.
\\
HC\cite{hutchinson1971, hutchinson1972}
&
$y = (1 + s \, t)^{1/s}$.
\\
MS\cite{martynov1983}
&
$y = \exp\left( \sqrt{1+2 \, t} - 1 \right)$.
\\
BPGG\cite{ballone1986}
&
$y = \exp\left[ (1 + s\,t)^{1/s} - 1\right]$.
\\
Verlet\cite{verlet1980}
&
$y = \exp\left[ t - A \, t^2/(2 + B \, t) \right]$.
\\
MP\cite{marucho2007}$^\dagger$
&
$y = 1 + [\exp(\eta \, t) - 1]/\eta$.
\\
RY
&
$y = 1+[\exp(f_\mathrm{sw} \, t) - 1] / f_\mathrm{sw}$,
with $f_\mathrm{sw} = 1 - \exp(-\alpha \, r)$.
\\
Quadratic
&
$y = 1 + t + \eta \, t^2/2$.
\\ \hline
\multicolumn{2}{p{\linewidth}}{
$^\dagger$
$\eta$ here is equivalent to $1/(a+1)$ in the original paper\cite{
marucho2007}.
}
\\ \hline
%
\end{tabular}
\caption{
Integral equations based on the OZ relation.
}
\label{tab:closures}
\end{table}




\begin{table*}\tiny
\newcommand{\vsp}{-2mm}
\begin{tabular}{l l r r r r r r r r r l l l l l l l l l}
\hline
Integral equation
&
&
$\left. B_{4} \middle/ B_2^{3} \right.$
&
$\left. B_{5} \middle/ B_2^{4} \right.$
&
$\left. B_{6} \middle/ B_2^{5} \right.$
&
$\left. B_{7} \middle/ B_2^{6} \right.$
&
$\left. B_{8} \middle/ B_2^{7} \right.$
&
$\left. B_{9} \middle/ B_2^{8} \right.$
&
$\left. B_{10} \middle/ B_2^{9} \right.$
&
$\left. B_{11} \middle/ B_2^{10} \right.$
&
$\left. B_{12} \middle/ B_2^{11} \right.$
\\
\hline
YBG\cite{
yvon1935, *born1946, nijboer1952, hill, *rice1965, hansen}
&
$\begin{aligned}
&c \\[\vsp]
&v \\[\vsp]
&y
\end{aligned}$
&
$\begin{aligned}
&0.34241 \\[\vsp]
&0.22522 \\[\vsp]
&0.33784
\end{aligned}$
&
$\begin{aligned}
&0.13350 \\[\vsp]
&0.04747 \\[\vsp]
&0.07595
\end{aligned}$
&
$\begin{aligned}
 0.031852& \\[\vsp]
-0.006357& \\[\vsp]
-0.010595&
\end{aligned}$
&
$\begin{aligned}
&-0.021721 \\[\vsp]
&-0.007000 \\[\vsp]
&-0.012001
\end{aligned}$
&
$\begin{aligned}
&-2.874\times10^{-2} \\[\vsp]
&-4.118\times10^{-3} \\[\vsp]
&-7.207\times10^{-3}
\end{aligned}$
&
$\begin{aligned}
&-2.907\times10^{-2} \\[\vsp]
&-1.206\times10^{-4} \\[\vsp]
&-2.143\times10^{-4}
\end{aligned}$
&
$\begin{aligned}
&-1.504\times10^{-2} \\[\vsp]
&-6.823\times10^{-4} \\[\vsp]
&-1.228\times10^{-3}
\end{aligned}$
&
$\begin{aligned}
-1.074\times10^{-2}& \\[\vsp]
 5.239\times10^{-4}&\\[\vsp]
 9.526\times10^{-4}&
\end{aligned}$
&
$\begin{aligned}
&-6.654\times10^{-4} \\[\vsp]
&-5.123\times10^{-4} \\[\vsp]
&-9.393\times10^{-4}
\end{aligned}$
\\
\hline
Kirkwood\cite{kirkwood1935, hill, *rice1965, hoover1962, stell1962}
&
$\begin{aligned}
&c \\[\vsp]
&v \\[\vsp]
&y
\end{aligned}$
&
$\begin{aligned}
&0.44182 \\[\vsp]
&0.13996 \\[\vsp]
&0.20919
\end{aligned}$
&
$\begin{aligned}
&0.28254 \\[\vsp]
&0.06556 \\[\vsp]
&0.11144
\end{aligned}$
&
$\begin{aligned}
&0.220338 \\[\vsp]
&0.026005 \\[\vsp]
&0.061075
\end{aligned}$
&
$\begin{aligned}
&0.145991 \\[\vsp]
&0.029434 \\[\vsp]
&0.050743
\end{aligned}$
&
$\begin{aligned}
&1.305\times10^{-1} \\[\vsp]
&9.369\times10^{-3} \\[\vsp]
&2.779\times10^{-2}
\end{aligned}$
&
$\begin{aligned}
&8.310\times10^{-2} \\[\vsp]
&1.670\times10^{-2} \\[\vsp]
&2.673\times10^{-2}
\end{aligned}$
&
$\begin{aligned}
&8.835\times10^{-2} \\[\vsp]
&9.215\times10^{-4} \\[\vsp]
&1.104\times10^{-2}
\end{aligned}$
&
$\begin{aligned}
&4.711\times10^{-2} \\[\vsp]
&1.175\times10^{-2} \\[\vsp]
&1.575\times10^{-2}
\end{aligned}$
&
$\begin{aligned}
 6.729\times10^{-2} & \\[\vsp]
-3.971\times10^{-3} & \\[\vsp]
 2.345\times10^{-3} &
\end{aligned}$
\\
\hline
PY\cite{percusyevick}
&
$\begin{aligned}
c& \\[\vsp]
v& \\[\vsp]
y&
\end{aligned}$
&
$\begin{aligned}
0.29688& \\[\vsp]
0.25000& \\[\vsp]
0.07813&
\end{aligned}$
&
$\begin{aligned}
0.12109& \\[\vsp]
0.08594& \\[\vsp]
-0.02188&
\end{aligned}$
&
$\begin{aligned}
0.044922& \\[\vsp]
0.027344& \\[\vsp]
0.011068&
\end{aligned}$
&
$\begin{aligned}
0.015625& \\[\vsp]
0.008301& \\[\vsp]
-0.004325&
\end{aligned}$
&
$\begin{aligned}
5.188\times10^{-3}& \\[\vsp]
2.441\times10^{-3}& \\[\vsp]
1.984\times10^{-3}&
\end{aligned}$
&
$\begin{aligned}
1.663\times10^{-3}& \\[\vsp]
7.019\times10^{-4}& \\[\vsp]
-8.613\times10^{-4}&
\end{aligned}$
&
$\begin{aligned}
5.188\times10^{-4}& \\[\vsp]
1.984\times10^{-4}& \\[\vsp]
3.922\times10^{-4}&
\end{aligned}$
&
$\begin{aligned}
1.583\times10^{-4}& \\[\vsp]
5.531\times10^{-5}& \\[\vsp]
-1.772\times10^{-4}&
\end{aligned}$
&
$\begin{aligned}
4.745\times10^{-5}& \\[\vsp]
1.526\times10^{-5}& \\[\vsp]
8.146\times10^{-5}&
\end{aligned}$
\\
\hline
HNC\cite{
morita1958, *morita1959, morita1960,
vanleeuwen1959,
*meeron1960, *verlet1960, *green1960,
*rushbrooke1960}
&
$\begin{aligned}
c& \\[\vsp]
v& \\[\vsp]
y&
\end{aligned}$
&
$\begin{aligned}
0.20919& \\[\vsp]
0.44531& \\[\vsp]
0.89063&
\end{aligned}$
&
$\begin{aligned}
0.04927& \\[\vsp]
0.14475& \\[\vsp]
0.36744&
\end{aligned}$
&
$\begin{aligned}
0.028067& \\[\vsp]
0.038180& \\[\vsp]
0.154134&
\end{aligned}$
&
$\begin{aligned}
0.006113& \\[\vsp]
0.044086& \\[\vsp]
0.090725&
\end{aligned}$
&
$\begin{aligned}
-1.459\times10^{-3}& \\[\vsp]
 6.461\times10^{-3}& \\[\vsp]
 1.077\times10^{-2}&
\end{aligned}$
&
$\begin{aligned}
4.033\times10^{-3}& \\[\vsp]
-8.554\times10^{-3}& \\[\vsp]
3.242\times10^{-3}&
\end{aligned}$
&
$\begin{aligned}
-1.010\times10^{-3}& \\[\vsp]
1.127\times10^{-2}& \\[\vsp]
1.800\times10^{-2}&
\end{aligned}$
&
$\begin{aligned}
-1.563\times10^{-3}& \\[\vsp]
-1.949\times10^{-3}& \\[\vsp]
-1.030\times10^{-2}&
\end{aligned}$
&
$\begin{aligned}
2.209\times10^{-3}& \\[\vsp]
-7.102\times10^{-3}& \\[\vsp]
-3.095\times10^{-3}&
\end{aligned}$
\\
\hline
$\begin{aligned}
&\mathrm{Hurst\cite{hurst1965}} \\[\vsp]
&m = 0.41718
\end{aligned}$
&
$\begin{aligned}
c& \\[\vsp]
v&
\end{aligned}$
&
$\begin{aligned}
0.28235& \\[\vsp]
0.28235&
\end{aligned}$
&
$\begin{aligned}
0.11019& \\[\vsp]
0.09151&
\end{aligned}$
&
$\begin{aligned}
0.038585& \\[\vsp]
0.035220&
\end{aligned}$
&
$\begin{aligned}
0.012813& \\[\vsp]
0.009847&
\end{aligned}$
&
$\begin{aligned}
4.309\times10^{-3}& \\[\vsp]
3.271\times10^{-3}&
\end{aligned}$
&
$\begin{aligned}
1.209\times10^{-3}& \\[\vsp]
1.304\times10^{-3}&
\end{aligned}$
&
$\begin{aligned}
4.107\times10^{-4}& \\[\vsp]
-1.341\times10^{-4}&
\end{aligned}$
&
$\begin{aligned}
8.969\times10^{-5}& \\[\vsp]
2.582\times10^{-4}&
\end{aligned}$
&
$\begin{aligned}
2.133\times10^{-5}& \\[\vsp]
-6.910\times10^{-5}&
\end{aligned}$
\\
\hline
$\begin{aligned}
&\mathrm{Rowlinson\,1\cite{rowlinson1965}} \\[\vsp]
&\eta = 0.16564
\end{aligned}$
&
$\begin{aligned}
c& \\[\vsp]
v&
\end{aligned}$
&
$\begin{aligned}
0.28235& \\[\vsp]
0.28235&
\end{aligned}$
&
$\begin{aligned}
0.11324& \\[\vsp]
0.07881&
\end{aligned}$
&
$\begin{aligned}
0.041426& \\[\vsp]
0.031550&
\end{aligned}$
&
$\begin{aligned}
0.014013& \\[\vsp]
0.008704&
\end{aligned}$
&
$\begin{aligned}
4.942\times10^{-3}& \\[\vsp]
1.856\times10^{-3}&
\end{aligned}$
&
$\begin{aligned}
1.432\times10^{-3}& \\[\vsp]
1.610\times10^{-3}&
\end{aligned}$
&
$\begin{aligned}
4.915\times10^{-4}& \\[\vsp]
-4.017\times10^{-4}&
\end{aligned}$
&
$\begin{aligned}
1.432\times10^{-4}& \\[\vsp]
2.350\times10^{-4}&
\end{aligned}$
&
$\begin{aligned}
1.596\times10^{-5}& \\[\vsp]
3.024\times10^{-5}&
\end{aligned}$
&
$\begin{aligned}
c& \\[\vsp]
v&
\end{aligned}$
\\
\hline
$\begin{aligned}
&\mathrm{Rowlinson\,2\cite{rowlinson1966}} \\[\vsp]
&\eta = 0.16564
\end{aligned}$
&
$\begin{aligned}
c& \\[\vsp]
v&
\end{aligned}$
&
$\begin{aligned}
0.28235& \\[\vsp]
0.28235&
\end{aligned}$
&
$\begin{aligned}
0.10920& \\[\vsp]
0.09568&
\end{aligned}$
&
$\begin{aligned}
0.036963& \\[\vsp]
0.041070&
\end{aligned}$
&
$\begin{aligned}
0.011460& \\[\vsp]
0.014171&
\end{aligned}$
&
$\begin{aligned}
3.466\times10^{-3}& \\[\vsp]
5.922\times10^{-3}&
\end{aligned}$
&
$\begin{aligned}
7.813\times10^{-4}& \\[\vsp]
2.681\times10^{-3}&
\end{aligned}$
&
$\begin{aligned}
2.042\times10^{-4}& \\[\vsp]
5.186\times10^{-4}&
\end{aligned}$
&
$\begin{aligned}
9.288\times10^{-7}& \\[\vsp]
5.440\times10^{-4}&
\end{aligned}$
&
$\begin{aligned}
-1.031\times10^{-5}& \\[\vsp]
4.024\times10^{-5}&
\end{aligned}$
\\
\hline
$\begin{aligned}
&\mathrm{HC\cite{hutchinson1971, hutchinson1972}} \\[\vsp]
&s = 0.83436
\end{aligned}$
&
$\begin{aligned}
c& \\[\vsp]
v&
\end{aligned}$
&
$\begin{aligned}
0.28235& \\[\vsp]
0.28235&
\end{aligned}$
&
$\begin{aligned}
0.11189& \\[\vsp]
0.08443&
\end{aligned}$
&
$\begin{aligned}
0.040157& \\[\vsp]
0.033256&
\end{aligned}$
&
$\begin{aligned}
0.013485& \\[\vsp]
0.009078&
\end{aligned}$
&
$\begin{aligned}
4.659\times10^{-3}& \\[\vsp]
2.485\times10^{-3}&
\end{aligned}$
&
$\begin{aligned}
1.334\times10^{-3}& \\[\vsp]
1.457\times10^{-3}&
\end{aligned}$
&
$\begin{aligned}
4.576\times10^{-4}& \\[\vsp]
-3.026\times10^{-4}&
\end{aligned}$
&
$\begin{aligned}
1.177\times10^{-4}& \\[\vsp]
2.630\times10^{-4}&
\end{aligned}$
&
$\begin{aligned}
1.877\times10^{-5}& \\[\vsp]
-2.975\times10^{-5}&
\end{aligned}$
\\
\hline
MS\cite{martynov1983}
&
$\begin{aligned}
c& \\[\vsp]
v&
\end{aligned}$
&
$\begin{aligned}
0.29688& \\[\vsp]
0.25000&
\end{aligned}$
&
$\begin{aligned}
0.11135& \\[\vsp]
0.12663&
\end{aligned}$
&
$\begin{aligned}
0.038090& \\[\vsp]
0.023846&
\end{aligned}$
&
$\begin{aligned}
0.011537& \\[\vsp]
0.021918&
\end{aligned}$
&
$\begin{aligned}
4.055\times10^{-3}& \\[\vsp]
-3.723\times10^{-3}&
\end{aligned}$
&
$\begin{aligned}
8.826\times10^{-4}& \\[\vsp]
7.791\times10^{-3}&
\end{aligned}$
&
$\begin{aligned}
4.929\times10^{-4}& \\[\vsp]
-5.373\times10^{-3}&
\end{aligned}$
&
$\begin{aligned}
-7.903\times10^{-5}& \\[\vsp]
5.132\times10^{-3}&
\end{aligned}$
&
$\begin{aligned}
1.525\times10^{-4}& \\[\vsp]
-4.482\times10^{-3}&
\end{aligned}$
\\
\hline
$\begin{aligned}
&\mathrm{BPGG\cite{ballone1986}} \\[\vsp]
&s = 1.83436
\end{aligned}$
&
$\begin{aligned}
c& \\[\vsp]
v&
\end{aligned}$
&
$\begin{aligned}
0.28235& \\[\vsp]
0.28235&
\end{aligned}$
&
$\begin{aligned}
0.10376& \\[\vsp]
0.11838&
\end{aligned}$
&
$\begin{aligned}
0.033542& \\[\vsp]
0.036488&
\end{aligned}$
&
$\begin{aligned}
0.010794& \\[\vsp]
0.014865&
\end{aligned}$
&
$\begin{aligned}
3.426\times10^{-3}& \\[\vsp]
3.981\times10^{-3}&
\end{aligned}$
&
$\begin{aligned}
9.949\times10^{-4}& \\[\vsp]
1.749\times10^{-3}&
\end{aligned}$
&
$\begin{aligned}
2.638\times10^{-4}& \\[\vsp]
4.766\times10^{-5}&
\end{aligned}$
&
$\begin{aligned}
4.516\times10^{-5}& \\[\vsp]
2.452\times10^{-4}&
\end{aligned}$
&
$\begin{aligned}
2.045\times10^{-5}& \\[\vsp]
-7.377\times10^{-5}&
\end{aligned}$
\\
\hline
$\begin{aligned}
&\mathrm{Verlet\cite{verlet1980}} \\[\vsp]
&A = 0.83436 \\[\vsp]
&B = 1.54751
\end{aligned}$
&
$\begin{aligned}
c& \\[\vsp]
v&
\end{aligned}$
&
$\begin{aligned}
0.28235& \\[\vsp]
0.28235&
\end{aligned}$
&
$\begin{aligned}
0.10659& \\[\vsp]
0.10659&
\end{aligned}$
&
$\begin{aligned}
0.034177& \\[\vsp]
0.046552&
\end{aligned}$
&
$\begin{aligned}
0.010100& \\[\vsp]
0.015639&
\end{aligned}$
&
$\begin{aligned}
2.993\times10^{-3}& \\[\vsp]
5.844\times10^{-3}&
\end{aligned}$
&
$\begin{aligned}
7.601\times10^{-4}& \\[\vsp]
1.814\times10^{-3}&
\end{aligned}$
&
$\begin{aligned}
2.436\times10^{-4}& \\[\vsp]
2.341\times10^{-4}&
\end{aligned}$
&
$\begin{aligned}
2.875\times10^{-5}& \\[\vsp]
2.884\times10^{-4}&
\end{aligned}$
&
$\begin{aligned}
1.430\times10^{-5}& \\[\vsp]
-5.269\times10^{-5}&
\end{aligned}$
\\
\hline
$\begin{aligned}
&\mathrm{MP\cite{marucho2007}} \\[\vsp]
&\eta = 0.16564
\end{aligned}$
&
$\begin{aligned}
c& \\[\vsp]
v&
\end{aligned}$
&
$\begin{aligned}
0.28235& \\[\vsp]
0.28235&
\end{aligned}$
&
$\begin{aligned}
0.11054& \\[\vsp]
0.09005&
\end{aligned}$
&
$\begin{aligned}
0.038712& \\[\vsp]
0.036136&
\end{aligned}$
&
$\begin{aligned}
0.012713& \\[\vsp]
0.010472&
\end{aligned}$
&
$\begin{aligned}
4.243\times10^{-3}& \\[\vsp]
3.442\times10^{-3}&
\end{aligned}$
&
$\begin{aligned}
1.168\times10^{-3}& \\[\vsp]
1.559\times10^{-3}&
\end{aligned}$
&
$\begin{aligned}
3.886\times10^{-4}& \\[\vsp]
-1.293\times10^{-4}&
\end{aligned}$
&
$\begin{aligned}
8.324\times10^{-5}& \\[\vsp]
2.951\times10^{-4}&
\end{aligned}$
&
$\begin{aligned}
1.508\times10^{-5}& \\[\vsp]
-5.052\times10^{-5}&
\end{aligned}$
\\
\hline
$\begin{aligned}
&\mathrm{RY\cite{rogers1984}} \\[\vsp]
&\alpha = 0.17015
\end{aligned}$
&
$\begin{aligned}
c& \\[\vsp]
v&
\end{aligned}$
&
$\begin{aligned}
0.28056& \\[\vsp]
0.28056&
\end{aligned}$
&
$\begin{aligned}
0.11101& \\[\vsp]
0.08631&
\end{aligned}$
&
$\begin{aligned}
0.038375& \\[\vsp]
0.037204&
\end{aligned}$
&
$\begin{aligned}
0.012714& \\[\vsp]
0.009674&
\end{aligned}$
&
$\begin{aligned}
4.474\times10^{-3}& \\[\vsp]
3.444\times10^{-3}&
\end{aligned}$
&
$\begin{aligned}
1.002\times10^{-3}& \\[\vsp]
1.951\times10^{-3}&
\end{aligned}$
&
$\begin{aligned}
4.974\times10^{-4}& \\[\vsp]
-7.341\times10^{-4}&
\end{aligned}$
&
$\begin{aligned}
2.546\times10^{-5}& \\[\vsp]
6.736\times10^{-4}&
\end{aligned}$
&
$\begin{aligned}
1.021\times10^{-6}& \\[\vsp]
-2.314\times10^{-4}&
\end{aligned}$
\\
\hline
$\begin{aligned}
&\mathrm{Quadratic} \\[\vsp]
&\eta = 0.16564
\end{aligned}$
&
$\begin{aligned}
c& \\[\vsp]
v&
\end{aligned}$
&
$\begin{aligned}
0.28235& \\[\vsp]
0.28235&
\end{aligned}$
&
$\begin{aligned}
0.11081& \\[\vsp]
0.08894&
\end{aligned}$
&
$\begin{aligned}
0.039034& \\[\vsp]
0.035331&
\end{aligned}$
&
$\begin{aligned}
0.012917& \\[\vsp]
0.009974&
\end{aligned}$
&
$\begin{aligned}
4.359\times10^{-3}& \\[\vsp]
3.135\times10^{-3}&
\end{aligned}$
&
$\begin{aligned}
1.220\times10^{-3}& \\[\vsp]
1.466\times10^{-3}&
\end{aligned}$
&
$\begin{aligned}
 4.118\times10^{-4}& \\[\vsp]
-1.901\times10^{-4}&
\end{aligned}$
&
$\begin{aligned}
9.377\times10^{-5}& \\[\vsp]
2.780\times10^{-4}&
\end{aligned}$
&
$\begin{aligned}
1.760\times10^{-5}& \\[\vsp]
-5.299\times10^{-5}&
\end{aligned}$
&
\\
\hline
Reference$^\dagger$
&
&
$0.286950$
&
$0.110251$
&
$0.0388821$
&
$0.0130229$
&
$4.1825\times10^{-3}$
&
$1.3097\times10^{-3}$
&
$4.030\times10^{-4}$
&
$1.206\times10^{-4}$
&
$3.77(36)\times10^{-5}$
\\
\hline
\multicolumn{11}{p{\linewidth}}{
%
In each cell,
the numbers on the first, second, and third (if exists) rows
are from the compressibility ($c$),
virial ($v$), and cavity ($u$) routes,
respectively.
%
The cavity-route numbers
were generally inaccurate,
so we only list them for the first four equations.

All integral equations
produced the correct $B_3/B_2^2 = 5/8$
from the compressibility and virial routes,
but not necessarily from the cavity route.
%
The cavity-route values were
$5/6$ (YBG),
$5/8$ (Kirkwood),
$-1/12$ (PY, MS, RY),
$5/4$ (HNC),
and
$0.13752$ (the rest).

$^\dagger$
The reference values were from Ref. \cite{schultz2014},
except for $B_{12}$,
which have been improved
from the previous study\cite{zhang2014}.
%
A few other improved or new values were
$1.61(19)\times10^{-3}\,[2_{14}]$,
$1.250(22)\times10^{-4}\,[3_{11}]$,
$1.01(8)\times10^{-5}\,[4_{10}]$,
$1.05(38)\times10^{-6}\,[4_{11}]$,
$3.13(10)\times10^{-5}\,[5_{11}]$,
and
$-2.34(20)\times10^{-5}\,[5_{12}]$
(the numbers in ``$[\dots]$''
assume the format of $D_n$).
}
\\
\hline
\end{tabular}
\caption{
Virial coefficients of the three-dimensional hard-sphere fluid.
}
\label{tab:virhs3d}
\end{table*}





\subsection{\label{sec:hs_results}
Virial coefficients of the hard-sphere fluid}





We computed the virial coefficients of the hard-sphere fluid
in various dimensions
using the DSC and related equations
in Sec. \ref{sec:selfconsistent}.
%
The results are shown
in Figs. \ref{fig:ievircmp},
\ref{fig:ievirlow},
and
\ref{fig:ievirhigh},
as well as Table \ref{tab:vir128}.
%
The Mayer-sampling results\cite{zhang2014, schultz2014}
were used as the references.
%
Generally,
the DSC equation worked the best
in high dimensions,
but less so in lower dimensions.





\subsubsection{DSC equation}





We first compared the DSC equation under
Eq. \eqref{eq:Bc_eq_Bv}
with the PY and HNC ones,
as the former is an interpolation of the latter two.
%
As shown in Fig. \ref{fig:ievircmp},
the DSC equation was superior to the PY and HNC equations
in most cases.
%
The PY equation gave
reasonably accurate virial coefficients
$B_n^{(c)}$ and $B_n^{(v)}$ in $D = 2$,
but not so in high dimensions.
%
The $B_n^{(\mathrm{PY}, \chi)}$
from Eq. \eqref{eq:Bn_ddP}
was inferior in $D = 2$,
although its error approached those
of $B_n^{(c)}$ and $B_n^{(v)}$
in higher dimensions.
%
We omit the cavity-route $B_n^{(y)}$
from Eq. \eqref{eq:Bn_cavity}
in the PY case due to the poor accuracy.
%
For the HNC equation,
although $B_n^{(c)}$,
$B_n^{(v)}$, and $B_n^{(y)}$
were all erratic in low dimensions,
they improved quickly with $D$,
with $B_n^{(c)}$ being the best among the three.



The above comparison shows that the PY and HNC equations
worked best for low and high dimensions, respectively.
%
Thus, we expect the DSC equation to be closer
to the PY one in low dimensions,
but the HNC one in high dimensions.
%
This was indeed the case.
%
We found that the $\lambda_{n-2}$ in Eq. \eqref{eq:lambda}
increased with the dimension and order.
%
In very high dimensions,
$\lambda_{n-2}$ was close to $1.0$,
the HNC limit.



As shown in
Figs. \ref{fig:ievircmp}(e) and \ref{fig:ievircmp}(f),
the order-by-order DSC equation based on Eq. \eqref{eq:Bc_eq_Bv}
was more accurate than
the single-parameter $B_4$-SC Rowlinson 2 equation
(cf. Table \ref{tab:closures})
in high dimensions
(note that the former is reduced to the latter,
if $\lambda_{n-2} \equiv \eta$).





\begin{table*}\scriptsize
\newcommand{\vsp}{-2.5mm}
\begin{tabular}{l l l l l}
\hline
$D$
&
$\left. B_{32} \middle/ B_2^{31} \right.$
&
$\left. B_{64} \middle/ B_2^{63} \right.$
&
$\left. B_{96} \middle/ B_2^{95} \right.$
&
$\left. B_{128} \middle/ B_2^{127} \right.$
\\
\hline
10
&
$\begin{aligned}
&-1.108603771(2) \\[\vsp]
&-1.16(6)
\end{aligned}$
&
$\begin{aligned}
&-5.81033442(2)\times10^5 \\[\vsp]
&\,
\end{aligned}$
&
$\begin{aligned}
&-1.144362730(5)\times10^{12} \\[\vsp]
&\,
\end{aligned}$
&
$\begin{aligned}
&-3.91959367(2)\times10^{18} \\[\vsp]
&\,
\end{aligned}$
\\
\hline
11
&
$\begin{aligned}
&-1.310853792 \\[\vsp]
&-1.270(27)
\end{aligned}$
&
$\begin{aligned}
&-1.546364657\times10^5 \\[\vsp]
&\,
\end{aligned}$
&
$\begin{aligned}
&-7.854488089\times10^{12} \\[\vsp]
&\,
\end{aligned}$
&
$\begin{aligned}
&-7.370834712\times10^{19} \\[\vsp]
&\,
\end{aligned}$
\\
\hline
12
&
$\begin{aligned}
&-1.119249796(2) \\[\vsp]
&-1.1351(43)
\end{aligned}$
&
$\begin{aligned}
&-2.24709007(1)\times10^6 \\[\vsp]
&-2.36(13)\times10^6
\end{aligned}$
&
$\begin{aligned}
&-2.24623459(2)\times10^{13} \\[\vsp]
&\,
\end{aligned}$
&
$\begin{aligned}
&-4.41449757(5)\times10^{20} \\[\vsp]
&\,
\end{aligned}$
\\
\hline
13
&
$\begin{aligned}
&-0.7522183291 \\[\vsp]
&-0.7569(19)
\end{aligned}$
&
$\begin{aligned}
&-2.105998963\times10^6 \\[\vsp]
&-2.195(27)\times10^6
\end{aligned}$
&
$\begin{aligned}
&-3.410012696\times10^{13} \\[\vsp]
&\,
\end{aligned}$
&
$\begin{aligned}
&-1.155110104\times10^{21} \\[\vsp]
&\,
\end{aligned}$
\\
\hline
14
&
$\begin{aligned}
&-0.422622529(1) \\[\vsp]
&-0.42266(8)
\end{aligned}$
&
$\begin{aligned}
&-1.426782124(8)\times10^6 \\[\vsp]
&-1.445(13)\times10^6
\end{aligned}$
&
$\begin{aligned}
&-3.239990574(3)\times10^{13} \\[\vsp]
&-3.19(27)\times10^13
\end{aligned}$
&
$\begin{aligned}
&-1.63664941(2)\times10^{21} \\[\vsp]
&\,
\end{aligned}$
\\
\hline
15
&
$\begin{aligned}
&-0.2072205699 \\[\vsp]
&-0.20850(8)
\end{aligned}$
&
$\begin{aligned}
&-7.568079484\times10^5 \\[\vsp]
&-7.679(25)\times10^5
\end{aligned}$
&
$\begin{aligned}
&-2.161072905\times10^{13} \\[\vsp]
&-2.176(47)\times10^{13}\,
\end{aligned}$
&
$\begin{aligned}
&-1.458667826(6)\times10^{21} \\[\vsp]
&-1.77(15)\times10^{21}
\end{aligned}$
\\
\hline
16
&
$\begin{aligned}
&-0.0914914861(3) \\[\vsp]
&-0.09212(17)
\end{aligned}$
&
$\begin{aligned}
&-3.32763625(3)\times10^5 \\[\vsp]
&-3.358(13)\times10^5
\end{aligned}$
&
$\begin{aligned}
&-1.098618930(7)\times10^{13} \\[\vsp]
&-1.122(12)\times10^{13}\,
\end{aligned}$
&
$\begin{aligned}
&-9.1044817(5)\times10^{20} \\[\vsp]
&-9.54(33)\times10^{20}
\end{aligned}$
\\
\hline
17
&
$\begin{aligned}
&-0.03722612839 \\[\vsp]
&-0.037331(11)
\end{aligned}$
&
$\begin{aligned}
&-1.264478996\times10^5 \\[\vsp]
&-1.2731(19)\times10^5
\end{aligned}$
&
$\begin{aligned}
&-4.52029946(1)\times10^{12} \\[\vsp]
&-4.594(15)\times10^{12}\,
\end{aligned}$
&
$\begin{aligned}
&-4.30551416(4)\times10^{20} \\[\vsp]
&-4.347(46)\times10^{20}
\end{aligned}$
\\
\hline
18
&
$\begin{aligned}
&-0.0142026367(7) \\[\vsp]
&-0.014206(13)
\end{aligned}$
&
$\begin{aligned}
&-4.28336585(3)\times10^4 \\[\vsp]
&-4.309\times10^4
\end{aligned}$
&
$\begin{aligned}
&-1.57434952(9)\times10^{12} \\[\vsp]
&-1.588(6)\times10^{12}\,
\end{aligned}$
&
$\begin{aligned}
&-1.6360377(2)\times10^{20} \\[\vsp]
&-1.669(14)\times10^{20}
\end{aligned}$
\\
\hline
19
&
$\begin{aligned}
&-5.148274944\times10^{-3} \\[\vsp]
&-5.1554(10)\times10^{-3}
\end{aligned}$
&
$\begin{aligned}
&-1.324198556(2)\times10^4 \\[\vsp]
&-1.3280(10)\times10^4
\end{aligned}$
&
$\begin{aligned}
&-4.80195836(3)\times10^{11} \\[\vsp]
&-4.837(9)\times10^{11}\,
\end{aligned}$
&
$\begin{aligned}
&-5.22326065(2)\times10^{19} \\[\vsp]
&-5.263(23)\times10^{19}
\end{aligned}$
\\
\hline
20
&
$\begin{aligned}
&-1.79110161(1)\times10^{-3} \\[\vsp]
&-1.7923(7)\times10^{-3}
\end{aligned}$
&
$\begin{aligned}
&-3.8040586(2)\times10^3 \\[\vsp]
&-3.8212(35)\times10^3
\end{aligned}$
&
$\begin{aligned}
&-1.3166841(1)\times10^{11} \\[\vsp]
&-1.3229(26)\times10^{11}\,
\end{aligned}$
&
$\begin{aligned}
&-1.4500721(2)\times10^{19} \\[\vsp]
&-1.467(7)\times10^{19}
\end{aligned}$
\\
\hline
21
&
$\begin{aligned}
&-6.027766825(1)\times10^{-4} \\[\vsp]
&-6.0383(34)\times10^{-4}
\end{aligned}$
&
$\begin{aligned}
&-1.029817070(2)\times10^3 \\[\vsp]
&-1.0316(9)\times10^3
\end{aligned}$
&
$\begin{aligned}
&-3.31237887(3)\times10^{10} \\[\vsp]
&-3.3209(48)\times10^{10}\,
\end{aligned}$
&
$\begin{aligned}
&-3.5955034(1)\times10^{18} \\[\vsp]
&-3.620(12)\times10^{18}
\end{aligned}$
\\
\hline
22
&
$\begin{aligned}
&-6.027766825(1)\times10^{-4} \\[\vsp]
&-6.0383(34)\times10^{-4}
\end{aligned}$
&
$\begin{aligned}
&-265.63390(2) \\[\vsp]
&-265.90(17)
\end{aligned}$
&
$\begin{aligned}
&-7.768860(1)\times10^{9} \\[\vsp]
&-7.786(12)\times10^{9}
\end{aligned}$
&
$\begin{aligned}
&-8.132020(2)\times10^{17} \\[\vsp]
&-8.184(20)\times10^{17}
\end{aligned}$
\\
\hline
23
&
$\begin{aligned}
&-6.325876896(2)\times10^{-5} \\[\vsp]
&-6.3260(30)\times10^{-5}
\end{aligned}$
&
$\begin{aligned}
&-65.8601111(3) \\[\vsp]
&-65.83(5)
\end{aligned}$
&
$\begin{aligned}
&-1.72053969(3)\times10^{9} \\[\vsp]
&-1.7235(20)\times10^{9}
\end{aligned}$
&
$\begin{aligned}
&-1.70597411(9)\times10^{17} \\[\vsp]
&-1.7113(27)\times10^{17}
\end{aligned}$
\\
\hline
24
&
$\begin{aligned}
&-1.99007384(6)\times10^{-5} \\[\vsp]
&-1.9906(6)\times10^{-5}
\end{aligned}$
&
$\begin{aligned}
&-15.806042(1) \\[\vsp]
&-15.823(11)
\end{aligned}$
&
$\begin{aligned}
&-3.634870(1)\times10^{8} \\[\vsp]
&-3.6398(27)\times10^{8}
\end{aligned}$
&
$\begin{aligned}
&-3.364252(2)\times10^{16} \\[\vsp]
&-3.375(9)\times10^{16}
\end{aligned}$
\\
\hline
25
&
$\begin{aligned}
&-6.166835490(2)\times10^{-6} \\[\vsp]
&-6.1699(8)\times10^{-6}
\end{aligned}$
&
$\begin{aligned}
&-3.69268813(2) \\[\vsp]
&-3.6910(22)
\end{aligned}$
&
$\begin{aligned}
&-7.3858742(2)\times10^{7} \\[\vsp]
&-7.380(8)\times10^{7}
\end{aligned}$
&
$\begin{aligned}
&-6.3045134(6)\times10^{15} \\[\vsp]
&-6.295(10)\times10^{15}
\end{aligned}$
\\
\hline
26
&
$\begin{aligned}
&-1.88716746(8)\times10^{-6} \\[\vsp]
&-1.8878(6)\times10^{-6}
\end{aligned}$
&
$\begin{aligned}
&-0.84367386(7) \\[\vsp]
&-0.8437(5)
\end{aligned}$
&
$\begin{aligned}
&-1.4530871(4)\times10^{7} \\[\vsp]
&-1.4535(10)\times10^{7}
\end{aligned}$
&
$\begin{aligned}
&-1.1324015(1)\times10^{15} \\[\vsp]
&-1.1335(17)\times10^{15}
\end{aligned}$
\\
\hline
27
&
$\begin{aligned}
&-5.715054215(4)\times10^{-7} \\[\vsp]
&-5.7176(13)\times10^{-7}
\end{aligned}$
&
$\begin{aligned}
&-0.189211501(2) \\[\vsp]
&-0.18911(9)
\end{aligned}$
&
$\begin{aligned}
&-2.7831662(1)\times10^{6} \\[\vsp]
&-2.7857(18)\times10^{6}
\end{aligned}$
&
$\begin{aligned}
&-1.9641505(3)\times10^{14} \\[\vsp]
&-1.9637(24)\times10^{14}
\end{aligned}$
\\
\hline
28
&
$\begin{aligned}
&-1.71568201(9)\times10^{-7} \\[\vsp]
&-1.71537(25)\times10^{-7}
\end{aligned}$
&
$\begin{aligned}
&-0.041782545(4) \\[\vsp]
&-0.041795(18)
\end{aligned}$
&
$\begin{aligned}
&-5.212393(3)\times10^{5} \\[\vsp]
&-5.2179(20)\times10^{5}
\end{aligned}$
&
$\begin{aligned}
&-3.3063620(4)\times10^{13} \\[\vsp]
&-3.3059(37)\times10^{13}
\end{aligned}$
\\
\hline
29
&
$\begin{aligned}
&-5.112987071\times10^{-8} \\[\vsp]
&-5.1145(12)\times10^{-8}
\end{aligned}$
&
$\begin{aligned}
&-9.107961141(3)\times10^{-3} \\[\vsp]
&-9.1061(24)\times10^{-3}
\end{aligned}$
&
$\begin{aligned}
&-9.58217992(3)\times10^{4} \\[\vsp]
&-9.5815(43)\times10^{4}
\end{aligned}$
&
$\begin{aligned}
&-5.436905(1)\times10^{12} \\[\vsp]
&-5.4384(43)\times10^{12}
\end{aligned}$
\\
\hline
30
&
$\begin{aligned}
&-1.5144105(1)\times10^{-8} \\[\vsp]
&-1.51472(17)\times10^{-8}
\end{aligned}$
&
$\begin{aligned}
&-1.9639410(7)\times10^{-3} \\[\vsp]
&-1.9636(6)\times10^{-3}
\end{aligned}$
&
$\begin{aligned}
&-1.73361945(5)\times10^{4} \\[\vsp]
&-1.7345(12)\times10^{4}
\end{aligned}$
&
$\begin{aligned}
&-8.74338(2)\times10^{11} \\[\vsp]
&-8.746(11)\times10^{11}
\end{aligned}$
\\
\hline
\multicolumn{5}{p{\linewidth}}{
In each cell,
the upper and lower numbers
were obtained from the DSC integral equation,
Eq. \eqref{eq:Bc_eq_Bv},
and Mayer sampling\cite{zhang2014}
[Algorithm G3, with parameter $M = 1$
(or $M = 2$ for $D = 30$)],
respectively.
%
The former number was extrapolated to zero grid size (SM 1.B).
%
We only keep a maximum of 10 significant digits.
}
\\
\hline
\end{tabular}
\caption{
Virial coefficients of high-dimensional hard-sphere fluids.
}
\label{tab:vir128}
\end{table*}





\subsubsection{Improving the DSC equation}





The DSC equation can be improved in two ways.
%
We first discuss the modified DSC condition,
Eq. \eqref{eq:Bc_eq_Bv_kappa1},
versus the strict one, Eq. \eqref{eq:Bc_eq_Bv}.
%
The latter is the special case of $\kappa_n \equiv 0$.
%
In high dimensions,
it suffices to use Eq. \eqref{eq:Bc_eq_Bv},
as shown in Table \ref{tab:vir128}
and Figs. \ref{fig:ievirlow} and \ref{fig:ievirhigh},
and the accuracy improved with increasing $D$.
%
For $D = 2$, $3$, or $6$,
more accurate virial coefficients
can be achieved by Eq. \eqref{eq:Bc_eq_Bv_kappa1}
with nonzero $\kappa_n$,
as shown in
Figs. \eqref{fig:ievirlow}(a)
and \eqref{fig:ievirlow}(b).
%
For $D = 4$ or $5$,
however,
the DSC equation failed even with a nonzero $\kappa_n$.



Alternatively, in low dimensions,
one can switch to the $\lambda$-DSC equation
based on the density series of $\lambda$,
Eq. \eqref{eq:DSC_lambda}.
%
For $D = 5$,
the $\lambda$-DSC equation
yielded the correct signs of the first 12 virial coefficients,
as shown in Fig. \ref{fig:ievirlow}(b).
%
For $D = 6$,
the $\lambda$-DSC equation
also produced more accurate low-order virial coefficients
and density components of the direct correlation function,
$c_l(r)$, than the DSC equation.
%
As shown in Fig. \ref{fig:cmpcorl},
although the DSC equation improved over the PY and HNC equations
in all cases,
and the sign of $c_3(r)$ for $r > 1$ was corrected
by the nonzero $\kappa_n$,
the large oscillations in $c_4(r)$ and $c_5(r)$
remained uncaptured.
%
The $\lambda$-DSC equation,
on the other hand,
produced the qualitatively-correct correlation functions.
%
Unfortunately,
the $\lambda$-DSC equation
is less stable for higher orders or higher dimensions,
as discussed in SM 1.F.
%
Thus, the DSC equation is preferred in these cases.
%
In the future,
one may consider interpolations
of the DSC and $\lambda$-DSC equations
for better accuracy in all dimensions.





\begin{figure}[h]
  \makebox[\textwidth][c]{
    \includegraphics[angle=0, width=0.9\linewidth]{fig/ievircmp.pdf}
  }%
  \caption{
    \label{fig:ievircmp}
    Virial coefficients of hard-sphere fluid
    from the DSC, PY, and HNC equations compared to those
    from Mayer sampling (reference) in
    (a) $D = 2$, (b) $D = 7$, (c) $D = 10$,
    and (d) $D = 15$ dimensions.
    %
    Empty (thin) and filled (thick) symbols indicate
    positive and negative numbers, respectively.
    %
    In (e) and (f),
    the relative error
    $\epsilon(B_n) \equiv
    \left|
      \left.
        B_n
      \middle/
        B_n^{(\mathrm{ref})}
      \right. - 1
    \right|$
    from the DSC equation (diamonds)
    was compared with that from
    the $B_4$-SC equation (circles).
    %
    For $D = 10$ [panel (e)],
    the errors of the two appeared to be
    similar in high orders as they became
    comparable to that of the Mayer sampling results.
}
\end{figure}





\begin{figure}[h]
  \newcommand{\sz}{1.5mm}
  \makebox[\textwidth][c]{
    \includegraphics[angle=0, width=1.0\linewidth]{fig/ievirlow.pdf}
  }%
  \caption{\label{fig:ievirlow}
    Virial coefficients of the hard-sphere fluid
    from the DSC and $\lambda$-DSC integral equations
    (``$\diamond$'', ``$\triangle$'',
    ``\protect\tikz{\protect \draw (18: \sz) -- (90: \sz) -- (162: \sz) -- (234: \sz) -- (306: \sz) -- (18: \sz)}'',
    and $\times$;
    black dashed lines for the DSC equation
    with $\kappa_n = 0$,
    blue dotted lines for that with $\kappa_n \ne 0$,
    green dot-dashed lines for the $\lambda$-DSC equation)
    compared to those from Mayer sampling
    (reference,
    ``$\square$'', ``$\triangledown$'', ``$\ocircle$'',
    and $+$, red solid lines).
    (a) $D = 2$, $3$, and $4$;
    (b) $D = 5$, $6$, $7$, and $8$;
    (c) $D = 9$, $10$, and $11$;
    (d) $D = 12$, $13$, and $14$.
    Empty (thin) and filled (thick) symbols
    indicate positive and negative numbers, respectively.
  }
\end{figure}





\begin{figure}[h]
  \makebox[\textwidth][c]{
    \includegraphics[angle=0, width=1.0\linewidth]{fig/ievirhigh.pdf}
  }%
  \caption{\label{fig:ievirhigh}
    Virial coefficients of the hard-sphere fluid
    for dimensions $D = 15$ to $30$
    from the DSC integral equation with $\kappa_n \equiv 0$
    (``$\diamond$'' and ``$\triangle$'')
    compared to those from Mayer sampling
    (reference, ``$\square$'' and ``$\triangledown$'').
    Empty (thin) and filled (thick) symbols indicate
    positive and negative numbers, respectively.
  }
\end{figure}





\begin{figure}[h]
  \makebox[\textwidth][c]{
    \includegraphics[angle=0, width=1.0\linewidth]{fig/cmpcorl.pdf}
  }%
  \caption{\label{fig:cmpcorl}
    Density components $c_l(r)$
    of the direct correlation function of the sixth-dimensional
    hard-sphere fluid from
    the DSC, $\lambda$-DSC, PY, and HNC integral equations
    compared to those from Mayer sampling (reference).
    Note that $c_l(r)$ is discontinuous at $r = 1$:
    $c_l(r) = -t_l(r)$ for $r < 1$,
    and
    $c_l(r) = y_l(r) - t_l(r)$ for $r > 1$.
    The $r < 1$ results and its continuous extensions to $r > 1$
    (only for the Mayer sampling results, light red lines)
    use the left vertical scale;
    the $r > 1$ results and its continuous extension to $r < 1$
    (light red lines)
    use the right vertical scale.
  }
\end{figure}





\subsection{\label{sec:gaussian_results}
Virial coefficients of the Gaussian fluid}





We applied the DSC equation to Gaussian model\cite{
uhlenbeck1962,
adin1992, *baram1986, baram1991}.
%
The model has an artificial potential
such that the Mayer $f$-function
is given by $f(\vr) = -\exp(-\vr^2)$.
%
The main appeal of the model
is that it offers many exactly computable
virial coefficients\cite{
uhlenbeck1962, mcquarrie, baram1991}.
%
Such a calculation involves
an enumeration of all cluster integrals
that can be represented graphically
by biconnected diagrams\cite{
uhlenbeck1962, hansen}.
%
The number of these diagrams, however,
grows rapidly with the order, $n$.
%
We enumerated biconnected diagrams
using the program,
NAUTY\cite{mckay1981, *mckay2014},
and obtained the exact virial coefficients up to
the 12th order
(the previous literature\cite{baram1991}
went up to $n = 9$).
%
The exact values were used below as the references.



\begin{table*}\scriptsize
\newcommand{\vsp}{-2mm}
\begin{tabular}{l l l l l l l}
\hline
$D$
&
$\left. B_{9} \middle/ B_2^{8}\right.$
&
$\left. B_{10} \middle/ B_2^{9}\right.$
&
$\left. B_{11} \middle/ B_2^{10}\right.$
&
$\left. B_{12} \middle/ B_2^{11}\right.$
&
${\kappa_n}^\dagger$
&
Error
\\
\hline
1
&
$+0.2390494307$
&
$+0.3013640718$
&
$+0.09833412320$
&
$-0.2293412365$
&
&
(exact)
\\
\hline
2
&
$-0.05042432227$
&
$+0.01774803702$
&
$+0.03920410910$
&
$-0.01324404230$
&
&
(exact)
\\
\hline
3
&
$+0.01519760720$
&
$-0.01384965413$
&
$+1.334755917\times10^{�3}$
&
$+7.604327249\times10^{�3}$
&
&
(exact)
\\
\hline
4
&
$-0.01019577105$
&
$+8.798988664\times10^{�3}$
&
$-5.576767192\times10^{�3}$
&
$+2.318690261\times10^{�3}$
&
&
(exact)
\\
\hline
5
&
$+0.02276854866$
&
$-0.02141505098$
&
$+0.02119558966$
&
$-0.02202301229$
&
&
(exact)
\\
\hline
6
&
$\begin{aligned}
&+0.0310 \\[\vsp]
&+0.0322 \\[\vsp]
&+0.03181551125
\end{aligned}$
&
$\begin{aligned}
&-0.0366 \\[\vsp]
&-0.03780 \\[\vsp]
&-0.03787508135
\end{aligned}$
&
$\begin{aligned}
&+0.0451 \\[\vsp]
&+0.0463 \\[\vsp]
&+0.04698384160
\end{aligned}$
&
$\begin{aligned}
&-0.0576 \\[\vsp]
&-0.0588 \\[\vsp]
&-0.06032609709
\end{aligned}$
&
$\begin{aligned}
0 \\[\vsp]
(0.047)_{n \ge 8} \\[\vsp]
\,
\end{aligned}$
&
$\begin{aligned}
4.6\% \\[\vsp]
2.6\% \\[\vsp]
\mathrm{(exact)}
\end{aligned}$
\\
\hline
7
&
$\begin{aligned}
&+0.02083 \\[\vsp]
&+0.020983 \\[\vsp]
&+0.02097369343
\end{aligned}$
&
$\begin{aligned}
&-0.02625 \\[\vsp]
&-0.02644 \\[\vsp]
&-0.02648332518
\end{aligned}$
&
$\begin{aligned}
&+0.0345 \\[\vsp]
&+0.03471 \\[\vsp]
&+0.03483566714
\end{aligned}$
&
$\begin{aligned}
&-0.04686 \\[\vsp]
&-0.04716 \\[\vsp]
&-0.04739468609
\end{aligned}$
&
$\begin{aligned}
0 \\[\vsp]
(0.0085)_{n \ge 8} \\[\vsp]
\,
\end{aligned}$
&
$\begin{aligned}
1.12\% \\[\vsp]
0.50\% \\[\vsp]
\mathrm{(exact)}
\end{aligned}$
\\
\hline
8
&
$\begin{aligned}
&+0.010165 \\[\vsp]
&+0.010196 \\[\vsp]
&+0.01019017060
\end{aligned}$
&
$\begin{aligned}
&-0.01285 \\[\vsp]
&-0.012883 \\[\vsp]
&-0.01288281025
\end{aligned}$
&
$\begin{aligned}
&+0.01694 \\[\vsp]
&+0.016991 \\[\vsp]
&+0.01699842937
\end{aligned}$
&
$\begin{aligned}
&-0.02315 \\[\vsp]
&-0.023215 \\[\vsp]
&-0.02323226253
\end{aligned}$
&
$\begin{aligned}
0 \\[\vsp]
(0.00325)_{n \ge 8} \\[\vsp]
\,
\end{aligned}$
&
$\begin{aligned}
0.35\% \\[\vsp]
0.13\% \\[\vsp]
\mathrm{(exact)}
\end{aligned}$
\\
\hline
9
&
$\begin{aligned}
&+4.219\times 10^{-3} \\[\vsp]
&+4.2225\times 10^{-3} \\[\vsp]
&+4.222773149\times 10^{-3}
\end{aligned}$
&
$\begin{aligned}
&-5.197\times 10^{-3} \\[\vsp]
&-5.2012\times 10^{-3} \\[\vsp]
&-5.202299404\times 10^{-3}
\end{aligned}$
&
$\begin{aligned}
&+6.701\times 10^{-3} \\[\vsp]
&+6.707\times 10^{-3} \\[\vsp]
&+6.708746298\times 10^{-3}
\end{aligned}$
&
$\begin{aligned}
&-8.973\times 10^{-3} \\[\vsp]
&-8.980\times 10^{-3} \\[\vsp]
&-8.983225704\times 10^{-3}
\end{aligned}$
&
$\begin{aligned}
0 \\[\vsp]
(0.0000924)_{n \ge 7} \\[\vsp]
\,
\end{aligned}$
&
$\begin{aligned}
0.119\% \\[\vsp]
0.036\% \\[\vsp]
\mathrm{(exact)}
\end{aligned}$
\\
\hline
10
&
$\begin{aligned}
&+1.5991 \times 10^{-3} \\[\vsp]
&+1.59949 \times 10^{-3} \\[\vsp]
&+1.599635591 \times 10^{-3}
\end{aligned}$
&
$\begin{aligned}
&-1.8935 \times 10^{-3} \\[\vsp]
&-1.8940 \times 10^{-3} \\[\vsp]
&-1.894245463 \times 10^{-3}
\end{aligned}$
&
$\begin{aligned}
&+2.3557 \times 10^{-3} \\[\vsp]
&+2.3563 \times 10^{-3} \\[\vsp]
&+2.356666276 \times 10^{-3}
\end{aligned}$
&
$\begin{aligned}
&-3.0526 \times 10^{-3} \\[\vsp]
&-3.0534 \times 10^{-3} \\[\vsp]
&-3.053853001 \times 10^{-3}
\end{aligned}$
&
$\begin{aligned}
0 \\[\vsp]
(0.00029)_{n \ge 6} \\[\vsp]
\,
\end{aligned}$
&
$\begin{aligned}
0.041\% \\[\vsp]
0.014\% \\[\vsp]
\mathrm{(exact)}
\end{aligned}$
\\
\hline
\end{tabular}
\caption{
Virial coefficients of the Gaussian model.
}
\label{tab:gaussian}
\end{table*}





We computed the virial coefficients of
the Gaussian model using the DSC equation.
%
Particularly,
we wish to see whether the modified DSC condition,
Eq. \eqref{eq:Bc_eq_Bv_kappa1},
with a nonzero $\kappa_n$, improved over
the $\kappa_n = 0$ case [Eq. \eqref{eq:Bc_eq_Bv}].
%
In this model,
$\kappa_4$ should always be zero,
because it happens to yield
the correct functional form of
$w_2(r) \propto \exp(-r^2)$
and hence the correct $B_4$.
%
A nonzero $\kappa_n$
is needed only for $n\ge 5$,
as shown in Table \ref{tab:gaussian}.
%
We optimized $\kappa_n$
to minimize the largest deviation
from the first 12 known virial coefficients,
and then used for the next few coefficients.
%
As shown in Table \ref{tab:gaussian2},
the higher-order virial coefficients
computed with the optimized $\kappa_n$
were more accurate than those with $\kappa_n = 0$.
%
For $D\le5$,
however,
the DSC equation failed even with a nonzero $\kappa_n$.





\begin{table}\footnotesize
\newcommand{\vsp}{-4mm}
\begin{tabular}{ l | l | l | l | l | l | l | l }
\hline
$D_n$
&
$\left. B_n \middle/ B_2^{n-1} \right.$
&
$D_n$
&
$\left. B_n \middle/ B_2^{n-1} \right.$
&
$D_n$
&
$\left. B_n \middle/ B_2^{n-1} \right.$
&
$D_n$
&
$\left. B_n \middle/ B_2^{n-1} \right.$
\\
\hline
$8_{13}$
&
$\begin{aligned}
&+0.03258 \\[\vsp]
&+0.03267 \\[\vsp]
&+0.03270058460
\end{aligned}$
&
$8_{14}$
&
$\begin{aligned}
&-0.04702 \\[\vsp]
&-0.04714 \\[\vsp]
&-0.047190578
\end{aligned}$
&
$8_{15}$
&
$\begin{aligned}
&+0.0693 \\[\vsp]
&+0.06950 \\[\vsp]
&+0.0695748
\end{aligned}$
&
$8_{16}$
&
$\begin{aligned}
&-0.1041 \\[\vsp]
&-0.10439 \\[\vsp]
&-0.104499
\end{aligned}$
\\
\hline
$9_{13}$
&
$\begin{aligned}
&+0.012399 \\[\vsp]
&+0.012409 \\[\vsp]
&+0.01241389218
\end{aligned}$
&
$9_{14}$
&
$\begin{aligned}
&-0.01760 \\[\vsp]
&-0.017613 \\[\vsp]
&-0.01761993778
\end{aligned}$
&
$9_{15}$
&
$\begin{aligned}
&+0.02556 \\[\vsp]
&+0.025582 \\[\vsp]
&+0.0255911475
\end{aligned}$
&
$9_{16}$
&
$\begin{aligned}
&-0.03788 \\[\vsp]
&-0.037906 \\[\vsp]
&-0.037918635
\end{aligned}$
\\
\hline
$9_{17}$
&
$\begin{aligned}
&+0.05711 \\[\vsp]
&+0.057161 \\[\vsp]
&+0.05717734
\end{aligned}$
&
$9_{18}$
&
$\begin{aligned}
&-0.08747 \\[\vsp]
&-0.08754 \\[\vsp]
&-0.08756290
\end{aligned}$
&
$9_{19}$
&
$\begin{aligned}
&+0.13582 \\[\vsp]
&+0.13593 \\[\vsp]
&+0.1359577
\end{aligned}$
&
$9_{20}$
&
$\begin{aligned}
&-0.2135 \\[\vsp]
&-0.21369 \\[\vsp]
&-0.213724
\end{aligned}$
\\
\hline
$10_{13}$
&
$\begin{aligned}
&+4.0931 \times 10^{-3} \\[\vsp]
&+4.0943 \times 10^{-3} \\[\vsp]
&+4.094811210 \times 10^{-3}
\end{aligned}$
&
$10_{14}$
&
$\begin{aligned}
&-5.650 \times 10^{-3} \\[\vsp]
&-5.6518 \times 10^{-3} \\[\vsp]
&-5.652506747 \times 10^{-3}
\end{aligned}$
&
$10_{15}$
&
$\begin{aligned}
&+7.997 \times 10^{-3} \\[\vsp]
&+7.9976 \times 10^{-3} \\[\vsp]
&+8.000503367 \times 10^{-3}
\end{aligned}$
&
$10_{16}$
&
$\begin{aligned}
&-0.011569 \\[\vsp]
&-0.0115721 \\[\vsp]
&-0.01157316609
\end{aligned}$
\\
\hline
$10_{17}$
&
$\begin{aligned}
&+0.017059 \\[\vsp]
&+0.0170633 \\[\vsp]
&+0.0170645144
\end{aligned}$
&
$10_{18}$
&
$\begin{aligned}
&-0.022583 \\[\vsp]
&-0.0225898 \\[\vsp]
&-0.022591150
\end{aligned}$
&
$10_{19}$
&
$\begin{aligned}
&+0.038950 \\[\vsp]
&+0.0389607 \\[\vsp]
&+0.03896213
\end{aligned}$
&
$10_{20}$
&
$\begin{aligned}
&-0.06011 \\[\vsp]
&-0.0601272 \\[\vsp]
&-0.0601283
\end{aligned}$
\\
\hline
\multicolumn{8}{p{\linewidth}}{
Each cell contains the virial coefficients
from the DSC integral equation
with $\kappa_n = 0$ (first row),
or that with the nonzero $\kappa_n$,
as listed in Table \ref{tab:gaussian} (second row),
and the exact value (the third row,
obtained from summing biconnected graphs
with fewest numbers of edges,
followed by an extrapolation along
the number of edges
assuming a geometric series).
}
\\ \hline
\end{tabular}
\caption{
Higher-order virial coefficients of the Gaussian model.
}
\label{tab:gaussian2}
\end{table}





\section{Summary and discussions}




In summary,
we have shown an algorithm
with polynomial time scaling
(which differs from that by Shaul et al.\cite{shaul2011})
to compute the virial coefficients
from various integral equations.
%
Using series transformations
on an order-by-order basis
with the detailed self-consistent (DSC) criteria
for a constraint,
we accurately predicted
the virial coefficients of the high-dimensional
hard-sphere fluid and Gaussian model.
%
Despite the success,
the method met some difficulty in low dimensions.
%
In the future,
the method may be improved by choosing better closures
and/or better correction functions
for the cavity distribution function.
%
Alternatively,
one might consider to incorporate
other self-consistent conditions\cite{
hiroike1957, *lee1995, hoover1962},
or explicit bridge functions\cite{
kwak2005, shaul2011, perkyns1997, *perkyns2002}.




\section{Acknowledgements}





It is a pleasure to thank Drs. Y. Mei, K. Dyer, and J. Perkyns
for helpful discussions.
%
The authors gratefully acknowledge the financial support
of the National Science Foundation (CHE-1152876)
and the Robert A. Welch Foundation (H-0037).
%
The authors are also grateful for the support
for the computational time on the Stampede
(of the Texas Advanced Computing Center
at the University of Texas at Austin)\cite{stampede}
and SuperMIC
(of the Louisiana State University)\cite{supermic}
supercomputers (TG-MCA93S001).





\section{Supplemental Materials}


1. Technical details.

2. Thermodynamic quantities in integral equations.





\bibliography{liquid}





\end{document}

